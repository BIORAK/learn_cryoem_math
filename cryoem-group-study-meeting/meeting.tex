\documentclass[11pt, oneside]{article}   	% use "amsart" instead of "article" for AMSLaTeX format
\usepackage{geometry}                		% See geometry.pdf to learn the layout options. There are lots.
\geometry{letterpaper}                   		% ... or a4paper or a5paper or ... 
%\geometry{landscape}                		% Activate for rotated page geometry
%\usepackage[parfill]{parskip}    		% Activate to begin paragraphs with an empty line rather than an indent
\usepackage{graphicx}				% Use pdf, png, jpg, or eps§ with pdflatex; use eps in DVI mode
								% TeX will automatically convert eps --> pdf in pdflatex		
\usepackage{amssymb}
\usepackage{hyperref}
\usepackage{multirow}
\usepackage{ulem}
\usepackage{listings}
\usepackage{url}
\usepackage{longtable}

\lstset{frameround=fttt,language=Python,breaklines=true}


%SetFonts


\title{Online CryoEM Study Group}
\author{Geoffrey Woollard, Vancouver, Canada}
%\date{}							% Activate to display a given date or no date

\begin{document}
\maketitle

\tableofcontents



\pagebreak
\section{Dates and Topics}
Given the current distribution of our global audience (90\%+  in North America and Europe), we will have the time in the morning, starting around 9 AM Pacific (12 PM Eastern). Meetings are generally on Thursdays.

\begin{center}
\small
 \begin{longtable}{|| c c c p{90mm} ||} 
 \hline
 Date & Time & Stream & Topic \\ [0.5ex] 
 \hline\hline
Th 19 Nov 2020 & \tiny{9 AM PST} &  blur/sharpen & background math \& defocus phase contrast  \\ 
 \hline
\sout{Th 26 Nov 2020}  &   & & (cancelled for US Thanksgiving )    \\ 
 \hline
Th 3 Dec 2020 & \tiny{9 AM PST} & blur & Fourier transform   \\ 
 \hline
Th 10 Dec 2020 & \tiny{9 AM PST} & sharpen & Fourier transform   \\ 
 \hline
Th 17 Dec 2020 & \tiny{9 AM PST} & blur & convolution, sampling, Nyquist   \\ 
 \hline
Th 7 Jan 2021 & \tiny{9 AM PST} & sharpen & convolution, sampling, Nyquist   \\ 
 \hline
Th 14 Jan 2020 & \tiny{9 AM PST} & blur & phase-contrast in the EM   \\ 
 \hline
Th 21 Jan 2021 & \tiny{9 AM PST} & sharpen & phase-contrast in the EM   \\ 
 \hline
Th 28 Jan 2021 & \tiny{9 AM PST} & blur & image formation (forward model)   \\ 
 \hline
Th 11 Feb 2021 & \tiny{8 AM PST} & sharpen & image formation (forward model), multislice   \\ 
 \hline
Th 18 Feb 2021 & \tiny{8 AM PST} & sharpen &  2D expectation-maximization \\ 
 \hline
Th 25 Feb 2021 & \tiny{8 AM PST} & blur & coordinate systems \& rotations  \\ 
 \hline
 Th 4 March 2021 & \tiny{8 AM PST} & sharpen & coordinate systems \& rotations  \\ 
 \hline
Th 11 March 2021 & \tiny{7:30 AM PST} & blur & interpolation  \\ 
 \hline
 Th 18 March 2021 & \tiny{8 AM PST} & sharpen & interpolation  \\ 
 \hline
Th 25 March 2021 & \tiny{8 AM PST} & blur & \tiny{Guest lecture: D. Dynerman, 3D reconstruction via direct Fourier inversion}  \\ 
 \hline
Th 1 Apr 2021 & \tiny{8 AM PST} & sharpen & 3D reconstruction via  Fourier inversion  \\ 
 \hline
Th 8 Apr 2021 & \tiny{9 AM PST} & sharpen & variational autoencoders  \\ 
 \hline
Th 15 Apr 2021 & \tiny{8 AM PST} & sharpen & cryoDRGN (variational autoencoders) with Ellen Zhong  \\ 
 \hline
Th 22 Apr 2021 & \tiny{9 AM PST} & blur/sharpen & office hours  \\ 
 \hline
Th 29 Apr 2021 & \tiny{9 AM PST} & blur/sharpen & \tiny{Guest Lecture: J. Krieger, normal mode analysis \& ProDy}  \\ 
 \hline
Th 6 May 2021 & \tiny{9 AM PST} & sharpen & \tiny{Guest Lecture: J. Krieger, normal mode analysis \& ProDy} \\ 
 \hline
Th 13 May 2021 & \tiny{9 AM PST} & blur & \tiny{Guest: Yong Zi Tan. Master class: preferred orientation}  \\ 
 \hline
Th 20 May 2021 & \tiny{8:30 AM PST} & sharpen & \tiny{Guest: Philip Baldwin. Master class: preferred orientation}  \\ 
 \hline
Th 27 May 2021 & \tiny{9 AM PST} & blur & \tiny{Fourier transforms and reciprocal space for the beginner (G. Jensen's course)}  \\ 
 \hline
Th 3 June 2021 & \tiny{9 AM PST} & blur/sharpen & office hours  \\ 
 \hline
Th 10 June 2021 & \tiny{9 AM PST} & blur & \tiny{Fourier transforms and reciprocal space for the beginner (G. Jensen's course)}  \\
 \hline
Th 17 June 2021& \tiny{9 AM PST} & blur/sharpen & \tiny{Glaeser, Nogales \& Chiu (GNC), 1 Introduction and Overview (1.1, 1.2)}  \\ 
 \hline
Th 24 June 2021& \tiny{9 AM PST} & blur/sharpen & GNC, 2 Sample preparation  \\
 \hline
Th 1 July 2021& \tiny{9 AM PST} & blur/sharpen & Office hours \\ 
 \hline
Th 8 July 2021 & \tiny{9 AM PST} & sharpen & Guest: Ben Himes. Journal club: multisclice  \\ 
 \hline
Th 15 July 2021 & \tiny{9 AM PST} & blur/sharpen & Journal club: 3DVA \\
 \hline
Th 22 July 2021 & \tiny{9 AM PST} & blur/sharpen & 3DVA practical \\
 \hline
Th 29 July 2021 & \tiny{9 AM PST} & blur/sharpen & GNC, 3 Data collection (3.2, 3.3) \\
 \hline
Th 5 Aug 2021& \tiny{9 AM PST} & blur/sharpen & Office hours  \\ 
 \hline
Th 12 Aug 2021& \tiny{5 PM PDT} & blur/sharpen & \tiny{Guest: Radostin Danev (chapter author) GNC 3.4, Practical considerations: defocus, stigmation, coma-free illumination, and phase plates}  \\ 
 \hline
Th 19 Aug 2021& \tiny{9 AM PDT} & blur/sharpen & \tiny{GNC, 4 Data processing. 4.3 CTF estimation and image correction (restoration)} \\
 \hline
Th 26 Aug 2021& \tiny{9 AM PDT} & blur/sharpen &  \tiny{Guest: Basil Greber (chapter author) GNC, 4.4 Merging data from structurally homogeneous subsets)} \\ 
 \hline
Th 2 Sept 2021& \tiny{9 AM PDT} & blur/sharpen & Office hours  \\ 
 \hline
Th 9 Sept 2021&  & blur/sharpen & No meeting  \\ %Xiao-chen Bai
 \hline
Th 16 Sept 2021& \tiny{9 AM PDT} & blur/sharpen & \tiny{Guest: Alexis Rohou (chapter author) GNC, 4 Data processing. 4.7 B factors and map sharpening}  \\
 \hline
Th 23 Sept 2021& \tiny{9 AM PDT} & blur/sharpen & \tiny{GNC, 4 Data processing. 4.5 3D classification of structurally heterogeneous particles}  \\ %Xiao-chen Bai
 \hline
Th 30 Sept 2021& \tiny{9 AM PDT} & blur/sharpen & \tiny{GNC, 4 Data processing. 4.8 Optical aberrations and Ewald sphere curvature}  \\ 
 \hline
Th 7 Oct 2021& \tiny{9 AM PDT} & blur/sharpen & \tiny{GNC, 5 Map Validation. 5.2 Measures of resolution: FSC and local resolution}  \\ 
 \hline
Th 14 Oct 2021& \tiny{9 AM PDT} & blur/sharpen & \tiny{GNC, 5 Map Validation. 5.3 Recognizing the effect of bias and over-fitting}  \\ 
 \hline
Th 21 Oct 2021& \tiny{9 AM PDT} & blur/sharpen & Office hours  \\ 
 \hline
Th 28 Oct 2021 & \tiny{9 AM PDT} & blur/sharpen &\tiny{GNC, 5 Map Validation. 5.4 Estimates of alignment accuracy}  \\
 \hline
Th 4 Nov 2021& \tiny{9 AM PDT} & blur/sharpen & GNC, 6 Model building and validation (6.2, 6.3)  \\ 
 \hline
Th 11 Nov 2021 & \tiny{9 AM PDT} & blur/sharpen &\tiny{GNC, 6 Model building and validation. \newline 6.4 Quality evaluation of cryo-EM map-derived models}\\ 
 \hline
Th 18 Nov 2021& \tiny{9 AM PDT} & blur/sharpen &\tiny{GNC, 6 Model building and validation. \newline 6.5 How algorithms from crystallography are helping electron cryo-microscopy} \\ [1ex]  % note PST on the next week, vancouver switches to pst 7 nov 2021 https://time.is/Vancouver
 \hline
 ... 2021 ... & ... &  ... & ...  \\ 
 \hline
\end{longtable}
\end{center}

\pagebreak
\section{General Information}

Feel free to share this document and direct people to sign up at \url{https://forms.gle/BUeUW14vV4pyQbDDA} so I have the emails in one place. Online meeting links are emailed to those on this list. {\bf Please join the Slack group and ask questions there, rather than emailing me.}

\subsection{Audience and Streams}
Over 2/3 of the audience are grad students. There rest of the audience is a mix of long term staff (facility manager, seniour scientist, research associate, unspecified industry position, etc.), principal investigators (industry or academic), postdocs, and undergraduate. Meetings are labelled for the intended audience using the code names {\bf blur} and {\bf sharpen} , referring to map sharpening. Remember that both blurred and sharpened maps are important for being able to see what is going on and build a model in to the map. The goal of this study group is to have both intuition and be able to connect that back to the math, never missing the forest for the trees.

{\bf  Blur stream}: Beginners and intermediates with may have years of expertise doing sample prep, collecting data on the scope, making maps, and building models. However, many people with such expertise expressed a desire to go deeper into the fundamentals, and understand things better under the hood, and requested help to become more confident in the math. One person put it bluntly, "I am interested in being more than a button pusher."

{\bf Sharpen stream}: This stream can be for people with a high degree of expertise that are looking to sharpen their skills, and interact with other experts. For example: methods developers looking to make friends with each other, university instructors, facility managers who have a strong working knowledge but want to go deeper into the foundations and refresh, people with strong math and physics backgrounds that are fairly new but will catch on quickly, experts in crystallography that are switching over to cryoEM but that have a strong theoretical basis in the underlying math already.

\subsection{Pre-requisites}
This is a sort of lower bound, so that you don't show up and become lost or frustrated, especially in the blur stream. Not all of this background is needed. To learn about trilinear interpolation, you don't need to know anything about electron optics.

\subsubsection{Math}
You should be able to hit the ground running if you did alright in first year undergraduate calculus or did a second and third year statistics/physics/chemistry course with a quantitative and computational emphasis. If it's been a while you may want to quickly run through some online videos (khan academy for example) and brush up. You don't have to be strong in pure math (real analysis, proofs)
\begin{enumerate}
	\item calculus (single and multiple integrals, derivatives)
	\item trigonometry (sin, cos, etc.)
	\item linear algebra, 2D and 3D cartesian geometry, vectors, matrices
	\item statistics and probability (expectation, mean/variance, random variables, common distributions like gaussian/uniform/poisson)
	\item summation notation (sigma sums)
\end{enumerate}

Here are some resources from \url{khanacademy.org} for brushing up on your math

\begin{enumerate}
	\item \href{https://www.khanacademy.org/math/differential-calculus/dc-diff-intro}{Unit: Derivatives: definition and basic rules} 
	\item \href{https://www.khanacademy.org/math/differential-calculus/dc-chain}{Unit: Derivatives: chain rule and other advanced topics}
	\item \href{https://www.khanacademy.org/math/integral-calculus/ic-integration} {Unit: Integrals}
	\item \href{https://www.khanacademy.org/math/integral-calculus/ic-series}{Unit: Series}
	\item \href{https://www.khanacademy.org/math/linear-algebra}{Linear Algebra}
	\item \href{https://www.khanacademy.org/math/statistics-probability/probability-library}{Unit: Probability}
	\item \href{https://www.khanacademy.org/math/statistics-probability/counting-permutations-and-combinations}{Unit: Counting, permutations, and combinations}
	\item \href{https://www.khanacademy.org/math/statistics-probability/random-variables-stats-library}{Unit: Random variables}
\end{enumerate}

There is also a great series from \href{https://www.youtube.com/c/3blue1brown}{3Blue1Brown} on \href{https://youtube.com/playlist?list=PLZHQObOWTQDPD3MizzM2xVFitgF8hE_ab}{The Essence of Linear Algebra}, focusing on visual intuition of  linear transformations; and the \href{https://youtube.com/playlist?list=PLZHQObOWTQDMsr9K-rj53DwVRMYO3t5Yr}{Essence of calculus}: "The goal here is to make calculus feel like something that you yourself could have discovered."

\subsubsection{Physics}
\begin{itemize}
	\item Some exposure to modern (20th century) physics in a first or second year undergraduate course (relativity, QM). This comes in especially for the forward model.
	\item Some intuition about what is physically happening in the sample, at room temperature, after vitrification, and during imaging. The 2017 Chemistry Nobel lectures by Henderson, Dubochet and Frank would be more than enough.
\end{itemize}

\subsection{Meeting Format}
The meetings are meant to be more informal that is typical in research talks. The point is to learn and discuss with other learners, experienced practitioners, and experts.
\begin{itemize}
	\item Lecture with Q\&A
	\item Flipped classroom: i.e. we have a syllabus with pre-reading (textbook chapter, review paper) that we go through before and then discuss online
	\item Office hours with someone available to answer questions.
\end{itemize}

\subsection{Slack}
We will use the Slack channel 'cryoem\_study\_group' for asynchronous chat. Please join the Slack group and ask questions there, rather than emailing me. You can request a link to join by emailing me.

\section{Learning Resources}

\begin{enumerate}
	\item A good place to start is Grant Jensen's \href{https://www.caltech.edu/about/news/grant-jensen-cryo-em}{popular} online course \href{https://jensenlab.caltech.edu/courses/}{'Getting Started in Cryo-EM' }.

	\item After getting up to speed on the prerequisites, another good next step is the content developed by Dr Frederick Sigworth and others at \url{https://cryoemprinciples.yale.edu/video-lectures}. If you are having problems with links, then try viewing his content on YouTube.

	\item  Also, I highly recommend the interactive learning material developed by \href{http://cryoem.tudelft.nl/group/arjen-jakobi/}{Arjen Jakobi}, for a course on \href{https://gitlab.tudelft.nl/aj-lab/teaching/-/wikis/NB4020}{High-Resolution Imaging at TUDelft}: "The practicals are computational assignments in the form of interactive Jupyter notebooks hosted in a virtual learning environment. These notebooks contain code that can be executed to perform certain tasks or visualize results; you do not need any active knowledge of coding to work through the notebook." For the curious, the code that generates the visualizations is available on the repository.
 
	\item I have made an annotated bibliography organized thematically \href{https://github.com/geoffwoollard/learn_cryoem_math#resources}{here}.

	\item Coding notebooks to play around with are \href{https://github.com/geoffwoollard/learn_cryoem_math/tree/master/nb}{here}. If there is incompatibility between the notebook and the \href{https://github.com/geoffwoollard/learn_cryoem_math/tree/master/src}{code base} in the repo, that is because the code base has been updated. Older version of the code are available via past commits.
\end{enumerate}

\pagebreak
\section{Upcoming Meetings}

\pagebreak

\subsection{16 Sept 2021 - blur/sharpen - Guest: Alexis Rohou (chapter author) Glaeser, Nogales \& Chiu (GNC), 4 Data Processing (4.7)}
\textendash {Pre-reading}
\begin{itemize}
	\item Alexis Rohou. (2021). 4.7 B factors and map sharpening 
\end{itemize}
\textendash {Questions}
\begin{enumerate}
	\item 4.7.1: Spend some time studying Figure 4.25. Why does the curve attenuate with a B-factor vs ideal? What is happening in the Fourier transform of the map, when the noise floor is reached (at $\sim 3.0,2.1 \AA$ in this figure)? Why is the noise floor constant? Does the attenuation from the B-factor necessarily imply a loss of signal? Why or why not?
	\item 4.7.4: What are some reasons that the intensity of the radial averaged amplitude spectrum would be lower at high resolution, compared with an ideal case. What exactly would the ideal case represent? For a particular specimen under consideration, how do we compute the ideal radial averaged amplitude spectrum?
	\item 4.7.5: Sketch the radial amplitude profiles (cf. Fig 4.25) corresponding to the different trends in Fig 4.27.
	\item 4.7.5: If there are non-linearities in a ResLog ($\AA^{-2} vs \log N$ plot, what might be the cause?
	\item 4.7.6: What happens in "over sharpening", and how can it be prevented?
	\item 4.7.7: Look up one of the references in the paragraph that starts with "Several algorithms have been proposed to overcome these limitations". After studying this chapter, see if you can follow the logic of the algorithm.
	\item 4.7.7: Discuss why the B-factor as estimated from the amplitude spectrum vs a ResLog plot would be different.
\end{enumerate}

\subsection{23 Sept 2021 - blur - Glaeser, Nogales \& Chiu (GNC), 4 Data Processing (4.5)}
\textendash {Pre-reading}
\begin{itemize}
	\item Xiao-chen Bai. (2021). 4.5 3D classification of structurally heterogeneous particles
\end{itemize}
\textendash {Questions}
\begin{enumerate}
	\item 4.5.2: Consider the following statement "the user is required to provide the number of possible structures in the data, which is difficult to predict in most cases" Do you agree? Why or why not? If you have some idea of "the number of possible structures in the data", but use the "wrong number", what might happen?
	\item 4.5.2: How much junk do you typically have in your sample? How much can you remove during data processing?
	\item 4.5.3: Draw a cartoon representing how particle subtraction is done. 
	\item 4.5.3: Consider Figure 4.18. What happens if the "rigid region" is kind of flexible?
	\item 4.5.4: What types of problems is symmetry expansion suitable for?
	\item 4.5.5: What is the difference between "masked 3D refinement" and "masked 3D classification"? When would we use one over the other?
	\item What new approaches were not covered in this chapter? What has your experience been like with them?
\end{enumerate}

%\subsection{Summer 2021}
%By this time the anticipated textbook by Wah Chiu, Robert Glaeser, and Eva Nogales will hopefully be out, and we can see how it looks. We could also plan some lectures.

%\subsection{20 May 2021 - Sharpen - Electron Optics: Lens Aberations }
%\textendash{Pre-reading}
%\begin{enumerate}
%	\item "2.3 Lens Abberations" (pp. 31-40) in Reimer, L., \& Kohl, H. (2008). Transmission Electron Microscopy Physics of Image Formation. Springer series in optical sciences (Vol. 51). http://doi.org/10.1007/978-0-387-34758-5
%	\end{enumerate}
%%\textendash {Questions}
%%\begin{enumerate}
%%	\item Install 
%%\end{enumerate}

%\subsection{3 June May 2021 - Sharpen - Electron Optics: Electron Waves and Phase Shifts }
%\textendash{Pre-reading}
%\begin{enumerate}
%	\item "3.1 Electron Waves and Phase Shifts" (pp. 45-55) in Reimer, L., \& Kohl, H. (2008). Transmission Electron Microscopy Physics of Image Formation. Springer series in optical sciences (Vol. 51). http://doi.org/10.1007/978-0-387-34758-5
%	\end{enumerate}
%%\textendash {Questions}
%%\begin{enumerate}
%%	\item Install 
%%\end{enumerate}


\pagebreak
\section{Past Meetings}
\pagebreak

\subsection{19 Nov 2020 - Background Math \& Defocus Phase Contrast}
\subsubsection{Blur}
\textendash {Pre-reading}
\begin{itemize}
\item \href{https://youtu.be/CZkPG95eoS0}{Complex numbers and the complex exponential} (10 min)
\item \href{https://cryoemprinciples.yale.edu/sites/default/files/files/1%20Review%20of%20Complex%20Numbers.pdf}{Review	of complex	numbers} (3 pages)
\item \href{https://youtu.be/m2Hm1ziZFZg}{Defocus phase contrast} (35 min)

\end{itemize}
\textendash {Questions}
\begin{itemize}
\item What is physically happening to the sample when the electron is detected at small diffraction angles vs large diffraction angles? What else can happen to the electron?
%Watch video 2.1 (10 min). Towards the end (7:40 mark) Dr. Sigworth makes some interesting 2D/3D plots showing the real and imaginary part of exp(i theta). Code up a way to plot this and share it with everyone. For example, you could publish a coding notebook on something like GitHub.
\end{itemize}

\subsubsection{Sharpen}
\textendash{Pre-reading}
\begin{itemize}
	\item \href{https://youtu.be/m2Hm1ziZFZg}{Defocus phase contrast} (35 min)
	\item 6.2 The Wave Equation for Fast Electrons in 'Advanced Computing in Electron Microscopy', Kirkland (2020), pp. 156-159.
\end{itemize}
\textendash {Questions}
\begin{itemize}
\item In Sigworth's derivation of $|\Psi|^2$ in the 'Defocus phase contrast' video, he made various assumptions such as small theta, small epsilon. Under what extreme conditions would they break down. Do these occur in cryoEM? In other experimental regimes besides what is typical in cryoEM?
\item Around the 23-25 min mark of the 'Defocus phase contrast' video the CTF seems to oscillate to zeros. Is this observed in practice? Why or why not?
\item Work through the derivation with pencil and eraser, justifying each step as best you can. come with your questions to the group study.
\item Biological samples are made of atoms that give faint contrast, when compared to samples with higher atomic numbers. Where does the atomic number of the sample come into the equations presented here? 
\end{itemize}

\subsection{3 Dec 2020 - Blur - Fourier Transform}
\textendash {Pre-reading}
\begin{itemize}
	\item \href{https://youtu.be/27bNryOu84g}{The Fourier transform in one dimension} (35 min)
	\item \href{https://cryoemprinciples.yale.edu/sites/default/files/files/3%20Fourier1D.pdf}{The 1D Fourier Transform} (7 pages)
	\item Interactive coding notebook \href{https://gitlab.tudelft.nl/aj-lab/teaching/-/wikis/NB4020}{'Practical 1 - Fundamentals of Image Processing'} in the High-resolution imaging course at TUDelft. Note there are multiple notebooks: Introduction, Fourier Series, Frequency Spectrum, 2-D Fourier Analysis. See the links at the bottom. For this week, work through notebooks "ip\_basics\_part1", "ip\_basics\_part2", "ip\_basics\_part3".
\end{itemize}
\textendash {Questions}
\begin{itemize}
	\item Prove the linearity, scale, shift, convolution properties of the FT in 1D.
	\item Equation 4 in \href{https://cryoemprinciples.yale.edu/sites/default/files/files/3%20Fourier1D.pdf}{The 1D Fourier Transform} expresses how "a narrower function of x transforms into a broader function of u". Can you think of some examples of this in practice? Hint: what happens when you change magnification? 
	\item Assume you have a 128 \AA box size, with pixel size of 1 \AA. What is the spacing spacing between bins in Fourier space, if there are 64 bins in the negative direction and 64 bins in the positive direction? What length ranges (in units of \AA) do the first few and last few frequency bins correspond to?  How many frequency bins are between 10 and 5 \AA, versus 5 and 2.5 \AA? \\ Hint: $\langle..., [0,1/128), [1/128,2/128), ... , [\frac{128/2-2}{128}, \frac{128/2-1}{128}),  [\frac{128/2-1}{128}, \frac{128/2}{128})\rangle$.
\end{itemize}

\subsection{10 Dec 2020 - Sharpen - Fourier Transform}
\textendash {Pre-reading}
\begin{itemize}
\item \href{https://youtu.be/J1ViNmmQnd0}{The Fourier transform in two and three dimensions} (43 min)
	\item \href{https://cryoemprinciples.yale.edu/sites/default/files/files/4%20Fourier2D-3D.pdf}{2D and 3D Fourier transforms} (9 pages)
	\item Interactive coding notebook \href{https://gitlab.tudelft.nl/aj-lab/teaching/-/wikis/NB4020}{'Practical 1 - Fundamentals of Image Processing'} in the High-resolution imaging course at TUDelft. Note there are multiple notebooks: Frequency Spectrum, 2-D Fourier Analysis. See the links at the bottom. For this week, play with notebooks "ip\_basics\_part4", "ip\_basics\_part5".
	\item Optional
	\begin{itemize}
		\item \href{http://www.fftw.org/links.html}{Explanatory Material:} "Tutorials and introductions to Fourier transforms and FFTs, in no particular order." \url{fftw.org}
		\item \href{http://homepages.inf.ed.ac.uk/rbf/CVonline/LOCAL_COPIES/PIRODDI1/NUFT/NUFT.html}{Roberta Piroddi and Maria Petrou. Non-Uniform Fourier Transform: A Tutorial}
		\item \url{https://en.wikipedia.org/wiki/Non-uniform_discrete_Fourier_transform}
	\end{itemize}
\end{itemize}

\textendash{Questions}
\begin{itemize}
	\item Sample a simple function (e.g. $\exp$, $\sin$, gaussian) in 1D or 2D. Then use some library to compute the FFT. Then compute the DFT in your own implementation. How close is the error? Work our the solution analytically for the continuous case. What should the answer be at some discrete points according to the continuous case, and how is it different from what the FFT gave? What is the typical floating point error? What was the speedup of the FFT, and how does this compare to the theoretical limit?
	\item In practice, how long are FFTs taking (2D and 3D)? How does this compare with other computational bottlenecks like disk I/O and interpolation?
	\item What are some numerical issues or bottlenecks that can arise when using an implementation of the FFT in practice? What are some ways to overcome them?
\end{itemize}

\subsection{17 Dec 2020 - Blur - Convolution, Sampling, Nyquist}
\textendash {Pre-reading}
\begin{itemize}
	\item \href{https://youtu.be/_F-YDwY9X30}{Fourier transform: convolution, sampling and Nyquist} (37 min)
	\item Interactive coding notebook \href{https://gitlab.tudelft.nl/aj-lab/teaching/-/wikis/NB4020}{'Practical 1 - Fundamentals of Image Processing'} in the High-resolution imaging course at TUDelft. Note there are multiple notebooks: 2-D Fourier Analysis, Convolutions. See the links at the bottom. For this week, work through notebook "ip\_basics\_part6".
\end{itemize}
\textendash {Questions}
\begin{itemize}
	\item Solve the "homework problem" (without peeking at the answer!) in the first part of \href{https://youtu.be/_F-YDwY9X30}{Fourier transform: convolution, sampling and Nyquist}. If $g(x) = e^{-\pi(x-2)^2} - e^{-\pi(x+2)^2}$, what is $G(u)$?
	\item By the Convolution theorem $f(x)=g(x)*h(x) \Rightarrow F(u)=G(u)H(u)$. Since $F(u)=G(u)H(u)=H(u)G(u)$ we should have $f(x)=g(x)*h(x)=h(x)*g(x)$. Show explicitly the commutativity of the convolution operator: i.e., $g(x)*h(x)=\int dt \ g(t)h(x-t)=\int dt \ h(t)g(x-t)=h(x)*g(x)$
	\item Look at the animation in the notebook "ip\_basics\_part6". Write down the mathematical object that is being visualized, including integration bounds. What does one snapshot of the animation represent?
	\item When does convolution (physically or computationally) happen in cryoEM happen? Be specific when connecting things back to the math, e.g. what is being convolved, what is it's functional form, and how many dimensions are involved?
	\item When we convolve with a {\it broad/wide} gaussian function, what is happening in Fourier space? How does this relate to Fourier filtering? Hint, use the convolution theorem.
\end{itemize}

\subsection{7 Jan 2021 - Sharpen - Convolution, Sampling, Nyquist}
\textendash{Pre-reading}
\begin{itemize}
	\item \href{https://youtu.be/_F-YDwY9X30}{Fourier transform: convolution, sampling and Nyquist} (37 min)
	\item Interactive coding notebook \href{https://gitlab.tudelft.nl/aj-lab/teaching/-/wikis/NB4020}{'Practical 1 - Fundamentals of Image Processing'} in the High-resolution imaging course at TUDelft. Note there are multiple notebooks. For this week, work through notebook "ip\_basics\_part6".
\end{itemize}
\textendash {Questions}
\begin{itemize}
	\item Around the 16 min mark in \href{https://youtu.be/_F-YDwY9X30}{Fourier transform: convolution, sampling and Nyquist}, we see that the de-convoluted (recovered) signal blows up around the origin, where there is a delta function. From the math, why exactly did this happen? Is there a way to over come this blowing up effect?
	\item Let $f_a=e^{-ax^2} \ ; \ a>0$. Show that $f_1*(f_2*f_3) = (f_1*f_2)*f_3$. Note that $\int_{-\infty}^{+\infty} dx \ e^{-ax^2} = \sqrt{\pi/a}$
	\item The notebook "ip\_basics\_part6" has an interactive where the size of the kernel (Sobel, Gaussian, etc) can be changed. Convoluting with larger kernels could be a more expensive computation, but we could speed it up by doing this in Fourier space by the convolution theorem. If we take the FT of the image and kernel, there sizes to not match, so how can we multiply them element wise? How would you match the pixel sizes and do things in Fourier space to achieve the same result as the real space convolution?
	\item In the last part of "ip\_basics\_part6", the Fourier filtering versus the convolution filtering appear the same for Guassian blur, but different for box blur, and very different for sharpen. What might be causing this? Vary the kernel size and notice the different run times. Is doing things in Fourier space always quicker? 
	\item How can we increase the Nyquist frequency during data collection to get higher resolution information? What is the trade off? What should guide our choice of an optimal Nyquist, given our particular microscope and specimen?
	\item Code up a simple example illustrating the convolution theorem, where you also actually do the convolution in real space. How close is it to the answer where you did the multiplication in Fourier space? What was the speedup? Now try speeding up the calculation by of the convolution in real space by making the convolution kernel smaller. How good of an approximation is this? Use a 2D projection of a 3D density map and a meaningful kernel (e.g. low pass filter) to build your intuition in a useful way.
\end{itemize}

\subsection{14 Jan 2020 - Blur - Phase-contrast in the EM}
\textendash{Pre-reading}
\begin{itemize}
	\item \href{https://cryoemprinciples.yale.edu/sites/default/files/files/2%20Phase%20contrast.pdf}{Phase-contrast imaging in the EM'} (10 pages)
	\item Interactive coding notebook \href{https://gitlab.tudelft.nl/aj-lab/teaching/-/wikis/NB4020}{'Practical 1 - Fundamentals of Image Processing'} in the High-resolution imaging course at TUDelft. Note there are multiple notebooks. Go to the one on the CTF ('ip\_basics\_part7').
\end{itemize}
\textendash {Questions}
\begin{enumerate}
	\item See panel (b) in the \href{https://static5.olympus-lifescience.com/data/olympusmicro/primer/images/mtf/modulationfigure6.jpg?rev=480E}{'signal star'} image. What would this image look like if the CTF was applied (i.e. through convolution)?
	\item As defined in equation 11 of \href{https://cryoemprinciples.yale.edu/sites/default/files/files/2%20Phase%20contrast.pdf}{'Phase-contrast imaging in the EM'}, what happens when the B factor is negative versus positive?
	\item In practice, on the scope, the CTF is easier to see in an FFT if you are at high mag. Why? Can you relate this back to the math? How is magnification playing into the equation?
	\item Can you tell the difference between underfocus and overfocus by eye? Try putting what you see into words? Why do they look different?
	\item As defined in equation 11 in \href{https://cryoemprinciples.yale.edu/sites/default/files/files/2%20Phase%20contrast.pdf}{'Phase-contrast imaging in the EM'}, what physical behaviour is involved in this B factor? What other factors of B are there in cryoEM and what physical behaviour is involved in those B factors? Why are they all called B-factors?
	\item With reference to equation 11 in \href{https://cryoemprinciples.yale.edu/sites/default/files/files/2%20Phase%20contrast.pdf}{'Phase-contrast imaging in the EM'}: "B has units of nm$^2$ or $ \AA^2$ and is called the 'B-factor' or 'envelope factor'. Good cryo-EM images have B values of $50-100 \AA^2$, but even these values are not so ideal. At $B = 100  \AA^ 2$ spatial frequencies of $5 \AA^2$ are attenuated to 1/e of their original amplitude, and higher spatial frequencies are attenuated even more". Go to 'ip\_basics\_part7' and play with B. Is there agreement between the B factor explanation and the interactive notebook? Take a closer look at how the interactive notebook \href{https://gitlab.tudelft.nl/aj-lab/teaching/-/blob/master/binder/hri_practical01/shared/misc.py#L757}{codes up the B factor} and what pixel size is used/assumed.
	\item Play with the amplitude contrast. Do you 'want' more or less amplitude contrast? If the amplitude contrast is off, how will this effect CTF fitting? Can you think of a good reason why many CTF estimators do not fit amplitude contrast? If you are unsure as to the amplitude contrast, what should you do?
	\item Play around with 'plot\_ctf\_img\_interactive' in Question 4 of 'ip\_basics\_part7'. Turn on 'ctf\_multiply'. Can you see how the information is delocalized more when the defocus increases? How defocused can the image get? Where does the information delocalize to? When collecting data on a real micrograph (versus computationally applying the CTF to a synthetic image the size of a particle), what issues would you have?
\end{enumerate}

\subsection{21 Jan 2021 - Sharpen - Phase-contrast in the EM}
\textendash{Pre-reading}
\begin{itemize}
	\item  Philippsen et al. 2007. "The contrast-imaging function for tilted specimens" \url{https://www.sciencedirect.com/science/article/pii/S0304399106001525}
	\item Voortman et al. 2011. "A fast algorithm for computing and correcting the CTF for tilted, thick specimens in TEM" \url{https://www.sciencedirect.com/science/article/pii/S0304399111000878}
	\item Voortman et al. 2012. "Fast, spatially varying CTF correction in TEM" \url{https://pubmed.ncbi.nlm.nih.gov/22728402/}	
\end{itemize}
%\textendash {Questions}
%\begin{itemize}
%	\item 
%\end{itemize}

\subsection{28 Jan 2021 - Blur - Image Formation (Forward Model)}
\textendash{Pre-reading}
\begin{itemize}
	\item \href{https://youtu.be/tzv5c5K7MEk?t=4690}{NCCAT SPA short course 2020, Lecture 4: Algorithms and foundational math Part I \& 2, Fred Sigworth} (1:18:10 - 1:28:46, ~10 min)
	\end{itemize}
\textendash {Questions}
\begin{enumerate}
	\item How is real noise different from Gaussian additive noise? What makes it different? 
	\item  If particles come from different types of grids (thin continuous carbon, single layer graphene) can they be combined into one large dataset to solve a structure?
	\item Why don't researchers pool different datasets of the same particle (e.g. from different EMPIAR entries) to solve a high resolution structure with tens of millions of particles?
\end{enumerate}

\subsection{11 Feb 2021 - Sharpen - Image Formation (Forward Model), Multislice}
\textendash{Pre-reading}
\begin{enumerate}
	\item 'Theory' section in Koeck, P. J. B., \& Karshikoff, A. (2015). Limitations of the linear and the projection approximations in three-dimensional transmission electron microscopy of fully hydrated proteins. Journal of Microscopy, 259(3), 197-209. \url{http://doi.org/10.1111/jmi.12253}
	\item Section '2. Theory' in Vulovi\'{c}, M., Ravelli, R. B. G., van Vliet, L. J., Koster, A. J., Lazi\'{c}, I., L\"ocken, U., Rieger, B. (2013). Image formation modeling in cryo-electron microscopy. Journal of Structural Biology, 183(1), 19?32. \url{http://doi.org/10.1016/j.jsb.2013.05.008}
	\begin{itemize}
		\item Supplementary material associated with the article (18 pages)
	\end{itemize}
	\item 6.2 The Wave Equation for Fast Electrons in 'Advanced Computing in Electron Microscopy', Kirkland (2020), pp. 156-159.
	\item 6.4 The Multislice Solution in 'Advanced Computing in Electron Microscopy', Kirkland (2020), pp. 162-165.
\end{enumerate}

\textendash {Questions}
\begin{enumerate}
	\item Equation 7 in Vulovi\'{c} et al. shows the DQE/NTF being applied. In practice, the DQE/MTF/NTF is often plotted in 1D in papers. What would be the difference between doing the convolution of the DQE/NTF in 2D versus 1D?
	\item Equation 5 in Vulovi\'{c} et al. shows the 'complex CTF'. What is the difference between this and the $\sin$ CTF that is often used? When is it appropriate to use one versus the other treatment?
	\item Simpler models assume additive gaussian noise and pixel independence, for instance see \href{https://youtu.be/tzv5c5K7MEk?t=4690}{NCCAT SPA short course 2020, Lecture 4: Algorithms and foundational math Part I \& 2, Fred Sigworth} (1:18:10 - 1:28:46, ~10 min). Is this the case in Vulovi\'{c} et al.? If so, what sorts of models would break the pixel independence?
	\item For the sake of argument, let's say that our goal is to use more accurate forward models (multislice based, higher order series expansions phase and amplitude object), instead of projection. In practice, how would one go about doing 3D reconstruction with these more accurate forward models? What would be the computational bottlenecks in doing so?
	\item What is the difference between the 'weak approximation' and the 'projection assumption'? See the discussion in Koeck \& Karshikoff (2015), 'Multisclie and single projection simulations of image formation with amplitude and phase contrast', p 204. Go through the derivation in Koeck \& Karshikoff (2015), and point out where precisely the assumptions are made. 
\end{enumerate}

%\pagebreak
%\subsection{18 Feb 2021 - Blur - 2D Expectation-maximization}
%\textendash{Pre-reading}
%\begin{itemize}
%	\item \href{https://repository.upenn.edu/cgi/viewcontent.cgi?article=1665&context=physics_papers}{Nelson, P. C. (2019). Chapter 12 : Single Particle Reconstruction in Cryo-electron Microscopy. In Physical Models of Living Systems (pp. 305?325).}
%	\item Interactive coding notebook \href{https://gitlab.tudelft.nl/aj-lab/teaching/-/wikis/NB4020}{'Practical 2 - 2D/3D reconstruction'} in High-resolution imaging course at UTDelft.
%\end{itemize}
%\textendash {Questions}
%\begin{itemize}
%	
%\end{itemize}

\subsection{18 Feb 2021 - Sharpen - 2D Expectation-maximization}
\textendash{Pre-reading}
\begin{enumerate}
	\item Eddy, S. R. (2004). What is Bayesian statistics? Nature Biotechnology, 22(9), 1177?1178. \url{http://doi.org/10.1038/nbt0904-1177}
	\item \href{https://repository.upenn.edu/cgi/viewcontent.cgi?article=1665&context=physics_papers}{Nelson, P. C. (2019). Chapter 12 : Single Particle Reconstruction in Cryo-electron Microscopy. In Physical Models of Living Systems (pp. 305?325).}
	\item Interactive coding notebook \href{https://gitlab.tudelft.nl/aj-lab/teaching/-/wikis/NB4020}{'Practical 2 - 2D/3D reconstruction'} in the High-resolution imaging course at TUDelft.
\end{enumerate}
\textendash {Questions}
\begin{enumerate}
	\item The primer on Bayesian statistics by S. Eddy concludes: "Using Bayesian methods, we can instead integrate over varying degrees of uncertainty in different aspects of the analysis." What is uncertain in the different aspects of cryoEM analysis? What would be good priors for each of these aspects? In practice, are people taking into account the uncertainty of these parameters, or using point estimates? Why?
	\item Some popular cryoEM software does 2D classification in Fourier space. Why do these equations work in Fourier space? What would have to be changed?
	\item In Nelson's treatment, there was only one 2D class. To do multiple classes, what would have to be changed? What is happening during a typical 2D classification 'round'? What is constant during a round and what is updated between rounds?
	\item How does a prior affect the Log loss function? What functional forms of the priors are convenient for optimization, and what is their interpretation in terms of probability?
	\item It can be shown that iid Gaussian additive noise in real space is also Gaussian additive noise in Fourier space. See \href{https://dsp.stackexchange.com/questions/24170/what-are-the-statistics-of-the-discrete-fourier-transform-of-white-gaussian-nois}{this dsp.stackexchange.com post} for example. Try to generalize this somehow. What happens for iid Gaussian additive noise that is radially dependent in Fourier space, or pixel dependent in real space?
\end{enumerate}

\subsection{25 Feb 2021 - Blur - Coordinate Systems \& Rotations}
\textendash{Pre-reading}
\begin{enumerate}
	\item \href{https://youtu.be/FDyenWWlPdU}{Omega Open Course. 2017. Spherical Coordinate System (With 3D Animation)}
	\item \href{https://youtu.be/w79nZGzWMyI}{Andrew Dotson. 2018. Deriving Spherical Coordinates (For Physics Majors)}
	\item \href{https://accio.github.io/AMIDD/assets/2020/04/JRQuine-MathBiophysicsBook.pdf}{Quine, JR. 4.1.2 Two dimensions, 4.1.3 Three dimensions, 4.2 Complex form of a rotation in {\it  Mathematical Techniques in Structural Biology},  pp. 26-28}
	\item \url{https://en.wikipedia.org/wiki/Euler_angles#Rotation_matrix}
	\item \href{https://www.mecademic.com/en/how-is-orientation-in-space-represented-with-euler-angles}{How is Orientation in Space Represented with Euler Angles?}
	\item \href{http://www.close-range.com/docs/Computing_Euler_angles_from_a_rotation_matrix.pdf}{Gregory G. Slabaugh. Computing Euler angles from a rotation matrix}
	%\item (for reference, don't have to read): \href{https://www.geometrictools.com/Documentation/EulerAngles.pdf}{Euler Angle Formulas}
\end{enumerate}

\textendash {Questions}
\begin{enumerate}
	\item Problems 4, 6, 7, from \href{https://accio.github.io/AMIDD/assets/2020/04/JRQuine-MathBiophysicsBook.pdf}{Quine, JR. 4.5 Problems  in {\it  Mathematical Techniques in Structural Biology},  pp. 30-31}
	\item Implement the pseudo-code in Figure 1 of \href{http://www.close-range.com/docs/Computing_Euler_angles_from_a_rotation_matrix.pdf}{Gregory G. Slabaugh. Computing Euler angles from a rotation matrix}, and use it to compute the Euler angles from a rotation matrix.
\end{enumerate}

\subsection{4 March 2021 - Sharpen - Coordinate Systems \& Rotations}
\textendash{Pre-reading}
\begin{enumerate}
	\item \href{https://www.coursera.org/lecture/robotics-flight/axis-angle-representations-for-rotations-4hTtQ}{Axis/Angle Representations for Rotations}
	\item \href{https://accio.github.io/AMIDD/assets/2020/04/JRQuine-MathBiophysicsBook.pdf}{Quine, JR. 4. Orthogonal Transformations and Rotations in Mathematical Techniques in Structural Biology, pp. 25-31}
    	\item \href{https://youtu.be/d4EgbgTm0Bg}{Visualizing quaternions (4d numbers) with stereographic projection}
	\item \href{https://youtu.be/zjMuIxRvygQ}{Quaternions and 3d rotation, explained interactively}
	\item \href{https://eater.net/quaternions/}{Visualizing quaternions An explorable video series}
	\item \href{https://en.wikipedia.org/wiki/Quaternions_and_spatial_rotation}{Wiki: Quaternions and spatial rotation}
\end{enumerate}
\textendash {Questions}
\begin{enumerate}
	\item all problems from \href{https://accio.github.io/AMIDD/assets/2020/04/JRQuine-MathBiophysicsBook.pdf}{Quine, JR. 4.5 Problems in {\it Mathematical Techniques in Structural Biology},  pp. 30-31}
	\item Go to the \href{https://eater.net/quaternions/video/doublecover}{doublecover} interactive, and select the 4D tab on the bottom. Find the quaternion that corresponds to a rotation of 90 deg in the cw direction about the z axis.
	\item Quaterions can be used to describe the rotation of the point $ r= \langle r_x,r_y,r_x \rangle \in \mathbb{R}^3$ about angle $\theta$ in the direction of vector $\langle q_x,q_y,q_z \rangle$ as follows. $p_{rotated} = q*p*q^{-1}$, where $q = \langle \cos(\theta/2),q_x\sin(\theta/2),q_y\sin(\theta/2),q_j\sin(\theta/2)\rangle$ and \\ $q^{-1}=\langle \cos(-\theta/2),q_x\sin(-\theta/2),q_y\sin(-\theta/2),q_z\sin(-\theta/2) \rangle$ encode the rotation; $\vec r$ is represented as the quaternion $p=(0,r_x,r_y,r_z)$. Question: using pencil and paper, compute the rotation of the point $ r=\langle1,0,0\rangle$ about the z-axis by $\theta=180^{\circ}$ (ccw or cw, it's the same for this angle). Do you get $r=\langle-1,0,0\rangle$? You will have to use the quaternion multiplication table ($ij=k$, etc.), and know how to interconvert quaternions to cartesian 3D vectors.
	\item Study the sections \href{https://en.wikipedia.org/wiki/Quaternions_and_spatial_rotation#Conversion to and from the matrix representation}{Conversion to and from the matrix representation} and \href{https://en.wikipedia.org/wiki/Quaternions_and_spatial_rotation#Performance_comparisons}{Performance comparisons} on the wiki page for \href{https://en.wikipedia.org/wiki/Quaternions_and_spatial_rotation}{Quaternions and spatial rotation}. If we want to rotate many xyz points in space with a 3D rotation, what is the efficient way to do this? Taking into account interconversions between different rotation conventions (Euler angles, rotation matrices, axis-angle, and quaternions) how should we efficiently store rotations versus do rotations?
	\item Besides storing and doing rotations, we have to search over rotations. Wikipedia has a section on \href{https://en.wikipedia.org/wiki/Quaternions_and_spatial_rotation#Differentiation_with_respect_to_the_rotation_quaternion}{Differentiation with respect to the rotation quaternion}. Unpack the equation for $\frac{\partial {\bf p'}}{\partial {\bf q}}$. Would this equation ever be useful in cryoEM? When would we want to estimate the rotation by optimization, and how would we do so with this equation?
%	In some scenarios, it may be useful to compute the gradient of a rotation, wrt
%	
%	If one uses Euler angles to parametrize rotations in cryoEM reconstruction, one can find the alignment of a particle by minimizing (by following the gradient), for instance, the L2 loss of the raw particle and the predicted projection, wrt each Euler angle. If  
\end{enumerate}

\subsection{11 March 2021 - Blur - Interpolation}
\textendash{Pre-reading}
\begin{enumerate}
	\item \href{https://youtu.be/AqscP7rc8_M}{2016. Resizing Images - Computerphile} (bilinear and nearest-neighbour interpolation)
	\item \href{https://youtu.be/poY_nGzEEWM}{2016. Bicubic Interpolation - Computerphile}
	\item \url https://en.wikipedia.org/wiki/Interpolation
\end{enumerate}

\textendash {Questions}
\begin{enumerate}
	\item Go to the Wiki page for \href{https://en.wikipedia.org/wiki/Interpolation}{interpolation} and look at the pictures illustrating different interpolation schemes: nearest-neighbour, linear, polynomial, spline; and 1D vs 2D. How would these schemes differ in terms of "accuracy, cost, number of data points needed, and smoothness of the resulting interpolant function"?
	\item Take a look at \href{https://docs.scipy.org/doc/scipy/reference/generated/scipy.interpolate.griddata.html}{scipy.interpolate.griddata}. When we are interpolating in Fourier space in 3D, what is points, values, and xi? How big are they?
	\item Does interpolating with complex numbers cause any problems?
	\item Search through the code of some cryoEM software packages and find where they are doing interpolation. You could look for the forward model (simulating data) or for the interpolating of the 3D Fourier map before taking the inverse Fourier transform to get the map in real space. What are the dimensions? Can you tell if it is linear interpolation, or some other method?
\end{enumerate}

\subsection{18 March 2021 - Sharpen - Interpolation}
\textendash{Pre-reading}
\begin{enumerate}
	\item \href{https://www.paulinternet.nl/?page=bicubic}{Explanation and Java/C++ implementation of (bi)cubic interpolation}
	\item \url{https://en.wikipedia.org/wiki/Bicubic_interpolation}
	\item \href{https://web.archive.org/web/20051024202307/http://www.geovista.psu.edu/sites/geocomp99/Gc99/082/gc_082.htm}{Application of interpolation to elevation samples}
\end{enumerate}

\textendash {Questions}
\begin{enumerate} 
	\item Go through the \href{https://github.com/scipy/scipy/blob/v1.5.4/scipy/interpolate/interpnd.pyx}{source} of \href{https://docs.scipy.org/doc/scipy/reference/generated/scipy.interpolate.griddata.html}{scipy.interpolate.griddata} . What is the bottle neck for interpolation? You can try to read through the code, and also empirically benchmark things (e.g. sample a 2D or 3D Gaussian, which will give you real values, then compute the Gaussian function at these points, and interpolate on a cartesian grid at a specified granularity).
	\item Code up your own implementation of some sort of interpolation, that works in a similar way as \href{https://docs.scipy.org/doc/scipy/reference/generated/scipy.interpolate.griddata.html}{scipy.interpolate.griddata}, ie data point coordinates, data values (at those coordinates), and points at which to interpolate data. What is the bottle neck? How would you speed it up? 
	\item Investigate how sensitive \href{https://docs.scipy.org/doc/scipy/reference/generated/scipy.interpolate.griddata.html}{scipy.interpolate.griddata} is to noise, by sampling a function, corrupting it with noise, and seeing how smooth things have to be for the interpolation to work out well. Can you find a situation where there are pathologies? 
	\item How is \href{https://docs.scipy.org/doc/scipy/reference/generated/scipy.ndimage.map_coordinates.html#scipy.ndimage.map_coordinates}{scipy.ndimage.map\_coordinates} different from \href{https://docs.scipy.org/doc/scipy/reference/generated/scipy.interpolate.griddata.html}{scipy.interpolate.griddata}?
\end{enumerate}

\subsection{25 March 2021 - Blur - Guest lecture: Dr. David Dynerman, 3D Reconstruction via Direct Fourier Inversion.}
\href{https://www.linkedin.com/in/david-dynerman-6a4a47101/}{Dr. David Dynerman} is currently the Chief Scientific Officer at \href{https://www.thepublichealthco.com/}{The Public Health Company}. Prior to this he lectured on cryoEM at Berkeley, where he also worked with Eva Nogales (Molecular \& Cell Biology) and Bernd Sturmfels (Mathematics) to develop new algorithms for 2D $\rightarrow$ 3D reconstruction in cryo-electron microscopy. His lecture will explain direct Fourier inversion (2D $\rightarrow$ 3D reconstruction). \\ \\
\textendash{Audience} 
\\
This lecture will be structured so that beginners will (hopefully) learn some useful basics, while folks with more background will gain an orientation of the topic with road signs on where to dive in for more detail.
\\ \\
\textendash{Learning objectives}
\begin{enumerate}
  \item understand the basic principles of how direct Fourier inversion produces 3D reconstructions from 2D images
  \item understand the basics of two key technical aspects of Fourier inversion: Fourier interpolation and back projection
  \item understand what kinds of map artifacts can arise during direct Fourier inversion
\end{enumerate}
\textendash{Pre-reading}
\begin{enumerate}
	\item \href{https://youtu.be/INtehLmqfmA}{Part 2: 3-D Waves and Transforms - G. Jensen} (13 min)
	\item \href{https://youtu.be/_Ngybc0Rjo0}{Part 4: 3-D Reconstruction - G. Jensen} (11 min)
\end{enumerate}
\textendash {Questions}
\begin{enumerate}
	\item How could we visualize a single particle slice in a 3D map - either in real space or reciprocal space? How would this type of visualization help you? What other visualizations would be suitable?
	\item How does understanding Fourier interpolation and back projection help use data processing software, refine high quality maps, and solve structures?
	\item Do you think Fourier inversion will always be used in cryoEM data processing, or will other methods eventually replace it?
	\item What other research fields also use similar techniques (medical imaging, materials science, etc), and how much do practitioners in those fields understand about Fourier inversion?
	\item How much do you want/need to know about Fourier inversion, and what will help you build intuition?
\end{enumerate}
\textendash{Meeting Recording} \\
{\tiny \url{https://utoronto.zoom.us/rec/share/iUQlpdJvEwJ4Z-GU3bQjJzccyPSJg0iocKhN5RFwyHXMKkuIpR5I3alav3UKfk4T.hA7WqNfNCNPpbqN7}}

Access Passcode: 041\&NOnv+F

\subsection{1 Apr 2021 - Sharpen - 3D Reconstruction via Direct Fourier Inversion.}

\textendash{Learning objectives}
\begin{enumerate}
  \item understand implementation details of 2D $\rightarrow$ 3D Fourier inversion
  \item understand what kinds of map artifacts can arise during Fourier inversion
  \item play with Fourier inversion of simple shapes, to build intuition
\end{enumerate}
\textendash{Pre-reading}
\begin{enumerate}
	\item \href{https://www.projectrhea.org/rhea/index.php/ECE637_tomographic_reconstruction_fourier_slice_theorem_S13_mhossain}{The Bouman Lectures on Image Processing. A Slecture by Maliha Hossain. Subtopic 6: Fourier Slice Theorem. 2013} (notes and 25 min video)
 	\item \href{https://youtu.be/YIvTpW3IevI}{ASTRA Toolbox: Fourier Slice Theorem} (3 min)
	\item \href{https://youtu.be/DcmL1JGoiPs}{ASTRA Toolbox: Proof of Fourier Slice Theorem} (2 min)
	\item \href{https://youtu.be/pZ7JlXagT0w}{ASTRA Toolbox: Filtered Backprojection (FBP)} (4 min)
\end{enumerate}
\textendash {Questions}
\begin{enumerate}
	\item Go to \href{https://github.com/geoffwoollard/learn_cryoem_math/blob/master/nb/fourier_slice_2D_3D_with_trilinear.ipynb}{this coding notebook} on Fourier inversion. Play around with things to build your intuition. You can look into the code that is doing the Fourier transforms or interpolation to see what is "inside the black box". Are there any artefacts in the reconstruction? What happens where there are less particles, errors in the alignments?
	\item Let's say you are trying to do a reconstruction, and you see some artefacts that are difficult to interpret, but that might be caused by angular estimates, preferred orientations, contrast/SNR is not good enough - or a combination of all of those causal factors. How could you play around with a synthetic example to track down what is causing artefacts in your reconstruction?
\end{enumerate}

\textendash{Meeting Recording} \\
{\tiny \url{https://utoronto.zoom.us/rec/share/I-d6u1KSWACwZr0fwSBjOVGFHTs2DIUeNn2U1VhiNeLd-onFbd7Z3AQHeTq\_Imdb.taZbZeiA\_6tTUTZa}}

Access Passcode: \texttt{qdF\^{}I7.\^{}c9}

\subsection{8 Apr 2021 - Sharpen - Variational Autoencoders }
\textendash{Learning objectives}
\begin{enumerate}
  \item understand the theory of variational autoencoders, in order to intelligently use methods such as cryoDRGN
\end{enumerate}
\textendash{Pre-reading}
\begin{enumerate}
	\item \href{https://www.siarez.com/projects/variational-autoencoder}{Variational autoencoder interactive demos with deeplearn.js}
	\begin{enumerate}
		\item Doersch, C. (2016). Tutorial on Variational Autoencoders, 1-23. \url{https://arxiv.org/pdf/1606.05908.pdf} (23 pages)
	\end{enumerate}
	\item \href{https://www.jeremyjordan.me/variational-autoencoders/}{Jeremy Jordan - Variational autoencoders}
	\item \href{https://towardsdatascience.com/intuitively-understanding-variational-autoencoders-1bfe67eb5daf}{Intuitively Understanding Variational Autoencoders}
%	\item \href{https://youtu.be/_2PZxw4FzDU}{015 Jensen's inequality \& Kullback Leibler divergence} (10 min)
%	\item \href{https://www.youtube.com/watch?v=uaaqyVS9-rM&feature=youtu.be}{Ali Ghodsi, Lec : Deep Learning, Variational Autoencoder, Oct 12 2017 [Lect 6.2]} (1 hr)
%	\item \href{https://youtu.be/P78QYjWh5sM}{Deep Learning Lecture 14: Karol Gregor on Variational Autoencoders and Image Generation} (43 min)
%	\item \href{http://videolectures.net/deeplearning2015_courville_autoencoder_extension/}{Aaron Courville. 2015. Variational Autoencoder and Extensions. Deep Learning Summer School, Montreal.} (1.5 hrs)
\end{enumerate}
\textendash {Questions}
\begin{enumerate}
	\item Describe the "latent space" in your own words. What other ideas does it relate to in structural biology?
	\item Regarding the "reparameterization trick": what is being reparameterized and why did we need to do this "trick"?
	\item Go through the figures in the blog post \href{https://www.jeremyjordan.me/variational-autoencoders/}{"Jeremy Jordan - Variational autoencoders"} and explain the take home of each figure.
	\item Why do {\it variational} autoencoders (vs. just plain vanilla autoencoders) allow us to smoothly interpolate between observations?
	\item A variational autoencoder reconstructs its own input. I we are interested in reconstructing 3D structures from 2D  cryoEM observations, how can we use variational approaches to generate 3D models? Consider how concretely we would compute reconstruction loss
	\item Code up a variational autoencoder on some toy data. For example in \href{https://tiao.io/post/tutorial-on-variational-autoencoders-with-a-concise-keras-implementation/}{Keras} or \href{https://towardsdatascience.com/variational-autoencoder-demystified-with-pytorch-implementation-3a06bee395ed}{PyTorch}
\end{enumerate}

\subsection{15 Apr 2021 - Sharpen - Guest Lecture with Ellen Zhong (MIT), cryoDRGN \& Variational Autoencoders}
\textendash{Pre-reading}
\begin{enumerate}
	\item \href{https://www.notion.so/cryoDRGN-tutorial-b932c021cb2c415282f182048bac16ff}{cryoDRGN tutorial}
	\item \href{https://youtu.be/yft_qhErStg}{Ellen Zhong. 2020. CryoDRGN: Deep generative models for reconstructing heterogeneous protein structures from cryo-EM. SBGrid Consortium (40 min).}
	\item Zhong, E. D., Bepler, T., Berger, B., \& Davis, J. H. (2021). CryoDRGN: reconstruction of heterogeneous cryo-EM structures using neural networks. Nature Methods, 18(2), 176-185. \url{http://doi.org/10.1038/s41592-020-01049-4}
	\begin{enumerate}
		\item Bepler, T., Zhong, E. D., Kelley, K., Brignole, E., \& Berger, B. (2019). Explicitly disentangling image content from translation and rotation with spatial-VAE, (NeurIPS 2019).
		\item Zhong, E. D., Bepler, T., Davis, J. H., \& Berger, B. (2019). Reconstructing continuous distributions of 3D protein structure from cryo-EM images. ICLR 2020, 1-20. \url{https://arxiv.org/pdf/1909.05215.pdf}
	\end{enumerate}
\end{enumerate}
\textendash {Questions}
\begin{enumerate}
	\item Look through the "Log" from the command
	\begin{lstlisting}
cryodrgn train_vae data/128/particles.128.mrcs \
	--ctf data/ctf.pkl  \
	--poses data/poses.pkl  \
	--zdim 8  \
	-n 50  \
	--uninvert-data  \
	-o tutorial/00_vae128 > tutorial_00.log							
	\end{lstlisting}		
	Take a deep breath and go through it line by line calmly and patiently. What do you understand? What is confusing?	Using the Log output as a guide, can you draw a schematic picture of the architecture?
	\item The tutorial suggests 50 epochs for training. Can you learn anything from the other epochs? What would indicate that you should run more/less epochs?
	\item Consider the figure with the caption "kmeans20 volumes: vol\_000.mrc to vol\_019.mrc". What do these results tell you about the particles? What would be some next steps upon seeing these results?	
	\item Compare the location of the 20 density maps in PCA vs UMAP space. How are they the same and how are they different?
	\item The tutorial notes that "Principal component trajectories can also produce nonphysical "motions", e.g. if there is discrete variability in the structure." What evidence form cryoDRGN supports discrete vs continuous variability?	 What role can domain knowledge play here?
	\item The tutorial explains how to filter out junk. What other ways are there to filter out junk and how does cryoDRGN compare with these? When would you use one over another?
	\item Look at "rhe resulting 6 clusters" from the GMM. On the right panel, why are there particles scattered/delocalized? Carefully read how the GMM in the tutorial was done. It might help to look up UMAP (For the curious I would recommend this \href{https://youtu.be/nq6iPZVUxZU}{SciPy talk} by one of the original UMAP authors). Note that this is a sort of "trick question".	
	\item The tutorial reports that "in practice, we often find that junk particles are located as outliers in the distribution of latent encodings." Why do you think this happens? What rules of thumb should you use to decide how many junk particles you should set aside. What problems might this cause, and how could you check if these problems were occurring? 
	\item It's nice to have fast code, but day and week long experiments put things in perspective. What counts as "long" and "short" run times for data processing. How long are you willing to wait. Be honest, because this can help developers provide methods that people end up using in practice.
	\item Compare the density maps with junk at 128 pixels vs high resolution without junk at 256 pixels.
	\item Look at the high resolution "PC trajectories" (movies). In your own words, how would you describe this trajectory? What might the biological meaning of it be? Non-biological? What work would we have to do to untangle biological from non-biological conclusions?
	\item Look at the "plot of the learning curve (i.e. average loss per epoch)". What can we conclude from this graph? What else might it look like, and what should we do in that case?
	\item Let's say you are interested in determining to what extent conformations are discrete or continuous (e.g. an ion channel opening). Describe a workflow, including the generation of an initial reference (volume, pose, ctf), to investigate this. What steps would be involved, what would you look for at each step, and about how long would each step take.
	\item Look at how the reconstruction changes with the number of particles.  Use the lasso tool to select small and smaller subsets and compare by eye in a 3D viewer.
	\item "CryoDRGN's graph traversal algorithm builds a nearest neighbor graph between all the latent embeddings, and then performs dijkstra's algorithm to find the shortest path on the graph between the anchors nodes." What biological meaning might such a path have? What other features of a path would have biological (or should we say physical) meaning? How might we extend \href{https://youtu.be/GazC3A4OQTE}{Dijkstra's algorithm} to find other interesting paths?
	\item Let's assume that our sample was experiencing conformational heterogeneity in a defined way: with one degree of freedom, but with two Gaussian distributed clusters along this degree of freedom. For example a protein opening and closing, where there is a Gaussian distribution around the open and closed state, along the coordinate connecting the two. If we used cryoDRGN on such a dataset, what would our results look like? How would things change as the spread of the Gaussians increased so that they bled into each other?
	\item A cryoEM dataset is frozen in time, and represents a time average. Sometimes we are interested in inferring the temporal sequence of events, for instance consider the conformational states of \href{https://youtu.be/y-uuk4Pr2i8}{Kinesin}. But is a temporal sequence always obvious from different states? Imagine different scenarios that might be easier and harder to infer a temporal sequence from the type of data available in cryoEM, and the analysis offered by cryoDRGN.
%	\item Look through the "Example output log" from the command 
%	\begin{lstlisting}
%	analyze tutorial/00_vae128 49 --flip --Apix 3.275
%	\end{lstlisting}
%	Take a deep breath and go through it line by line calmly and patiently. What do you understand? What is confusing? Using the $\texttt{'model\_args'}$ as a guide, can you draw a schematic picture of the architecture?
\end{enumerate}
\textendash{Meeting Recording} \\
{\tiny \url{https://utoronto.zoom.us/rec/share/Zh_SoIGK_JzYBpWjMuOK_oVNIZeBkjBFhjuskbTq-fmxwK3JyIX8LHmG7O2xeo6j.fkEa4BQKud093XyX}}

Access Passcode: \texttt{4nyaZv8AW\#}

\subsection{22 Apr 2021 - Blur/Sharpen - Office Hours }
\textendash{Pre-reading}
\begin{enumerate}
	\item \href{https://youtu.be/eRPue0-Pkw4}{S2C2 CryoEM Image Processing Workshop: Day 1} (5 hrs). See schedule at the \href{https://youtu.be/eRPue0-Pkw4?t=570}{9:30 min mark}. Have a look at:
	\begin{enumerate}
		\item \href{https://www.youtube.com/watch?v=eRPue0-Pkw4&t=3666s}{1:01:06 Import movies}
		\item \href{https://www.youtube.com/watch?v=eRPue0-Pkw4&t=5455s}{1:30:55 Motion correction}
		\item \href{https://www.youtube.com/watch?v=eRPue0-Pkw4&t=7214s}{2:00:14 CTF estimation}
		\item \href{https://www.youtube.com/watch?v=eRPue0-Pkw4&t=8272s}{2:17:52 Exposure curation}
		\item \href{https://www.youtube.com/watch?v=eRPue0-Pkw4&t=9581s}{2:39:41 Particle picking}
		\item \href{https://www.youtube.com/watch?v=eRPue0-Pkw4&t=12790s}{3:33:10 2D classification}
	\end{enumerate}
\end{enumerate}
\textendash {Questions}
\begin{enumerate}
	\item Some data processing algorithms have many parameters, and it can be difficult to understand what exactly they are all doing, which ones go together, and when you would use them, how they would change things, and when you'd know that things are working. Find some parameters in a data processing algorithm that you don't understand, and bring it to the office hours to discuss. If you have time, run some jobs by changing a few parameters.
	\item Some people would like to know more about how an cryoEM algorithm works, but can't follow the original paper describing. Bring some results you've generated from some cryoEM data processing algorithm to discuss.
	\item Some data processing algorithms generate intermediate plots. If you know how to interpret these plots, they can help diagnose if the algorithm it working how you think it should be. Bring some plots that you have difficulty understanding to the office hours to discuss.  
	\item Some cryoEM discussion forums have "power users" that really understand things thoroughly. If you can follow how they are thinking, you can learn a lot, trouble shoot your data processing problems, and move your projects along. Find some posts and threads with these power users, and bring them to the office hours to discuss.
\end{enumerate}


\subsection{29 Apr 2021 - Blur - Guest Lecture with Dr. James Krieger (University of Pittsburgh), normal mode analysis \& ProDy}
\textendash{Pre-reading}
\begin{enumerate}
	\item \href{https://www.ncbi.nlm.nih.gov/pmc/articles/PMC2836427/}{Bahar, I., Lezon, T. R., Bakan, A., \& Shrivastava, I. H. (2010). Normal Mode Analysis of Biomolecular Structures: Functional Mechanisms of Membrane Proteins. Chemical Reviews, 110(3), 1463-1497. http://doi.org/10.1021/cr900095e}
	\item ProDy \href{http://prody.csb.pitt.edu/}{website}
\end{enumerate}
\textendash {Questions}
\begin{enumerate}
	\item Have a look at some \href{http://prody.csb.pitt.edu/tutorials/}{ProDy tutorials}
	\item In reality, each single particle image is a slightly different conformation. Let's assume that individual particles are fluctuations around an average reference state, and that the fluctuations are described well by the 'large breathing motions' described by normal mode approaches. How does this type of heterogeneity affect the final reconstruction? If heterogeneous particles go into one structure, what might happen to their alignment estimation?
	\item How can physics based models, the type ProDy uses, be incorporated into cryoEM data processing algorithms and downstream analysis?
\end{enumerate}

\textendash{Meeting Recording} \\
{\tiny \url{https://utoronto.zoom.us/rec/share/-nAL6ZajeveRYxzFGjtZZBd5DTlpCe1fDyT1xhtvDSGxxzh36gYMVOPJerrZGO3U.Iyo-PxJegZPkm7qH}}

Access Passcode: \texttt{@biL0D3=tw}

\subsection{6 May 2021 - Sharpen - Guest Lecture with Dr. James Krieger (University of Pittsburgh), normal mode analysis \& ProDy}
\textendash{Pre-reading}
\begin{enumerate}
	\item "2. Theory" section (pp. 1470-1477) in \href{https://www.ncbi.nlm.nih.gov/pmc/articles/PMC2836427/}{Bahar, I., Lezon, T. R., Bakan, A., \& Shrivastava, I. H. (2010). Normal Mode Analysis of Biomolecular Structures: Functional Mechanisms of Membrane Proteins. Chemical Reviews, 110(3), 1463-1497. http://doi.org/10.1021/cr900095e}
	\item Zhang, Y., Krieger, J., Mikulska-Ruminska, K., Kaynak, B., Sorzano, C. O. S., Carazo, J. M., ... Bahar, I. (2021). State-dependent sequential allostery exhibited by chaperonin TRiC/CCT revealed by network analysis of Cryo-EM maps. Progress in Biophysics and Molecular Biology, 160, 104-120. \url{http://doi.org/10.1016/j.pbiomolbio.2020.08.006}
	\item ProDy \href{https://github.com/prody/ProDy}{source}, especially \href{https://github.com/prody/ProDy/tree/master/prody/dynamics}{dynamics} and the \href{https://github.com/prody/ProDy/blob/master/prody/proteins/emdfile.py#L466}{Topology Representing Networks} class.
\end{enumerate}
\textendash {Questions}
\begin{enumerate}
	\item How can physics based models, the type ProDy uses, be incorporated into cryoEM data processing and analysis?
	\item What are the computational bottlenecks (runtime, memory) in the types of approaches in \href{https://github.com/prody/ProDy/tree/master/prody/dynamics}{prody.dynamics}?
	\item Look at the tutorial on \href{http://prody.csb.pitt.edu/tutorials/cryoem_tutorial/em_analysis.html#map-pseudo-atoms-to-pdb-atomic-model}{Processing Cryo-EM Electron Density Maps}. How many pseudo atoms were used to represent the map? Do you think this is reasonable? How many degrees of freedom to M pseudo atoms represent? How many degrees of freedom are needed to encode a map in an NxN real spaced grid? In a low pass Fourier space or Fourier cropped map (e.g. a $k_x$ by $k_y$ cube or sphere of radius $|k|$ pixels)?
\end{enumerate}


\subsection{13 May 2021 - Blur/Sharpen - Master class: preferred orientation. Yong Zi Tan (The Hospital for Sick Children, Toronto, Canada)}

\textendash{Learning Objective:}
\begin{itemize}
	\item Understand, diagnose, and solve the problem of preferred orientation.
\end{itemize}
\textendash{The Problem}
\begin{itemize}
	\item What is preferred orientation? Why do samples suffer from preferred orientation?
\begin{enumerate}
\item Blog post by Nanoimaging Services on Preferred Orientation (easy to understand) {\tiny\url{https://www.nanoimagingservices.com/blogs/post/solving-the-challenges-of-preferred-orientation-in-cryoem-sample-preparation?utm_source=Zoho&amp;utm_medium=social&amp;utm_campaign=May3}}
	\item Noble, A.J., Dandey, V.P., Wei, H., Brasch, J., Chase, J., Acharya, P., Tan, Y.Z., Zhang, Z., Kim, L.Y.,
Scapin, G. and Rapp, M., 2018. Routine single particle CryoEM sample and grid characterization by
tomography. Elife, 7, p.e34257. \url{https://elifesciences.org/articles/34257}
\end{enumerate}
\end{itemize}
\textendash{Diagnosing the Problem}
\begin{itemize}
	\item How do we know we have a preferred orientation problem? What is the consequence of it? How would we quantify the degree of preferred orientation?
	\begin{enumerate}
		\item Baldwin, P.R. and Lyumkis, D., 2020. Non-uniformity of projection distributions attenuates resolution
in Cryo-EM. Progress in biophysics and molecular biology, 150, pp.160-183. \url{https://www.sciencedirect.com/science/article/abs/pii/S0079610719301087}
		\item Tan, Y.Z., Baldwin, P.R., Davis, J.H., Williamson, J.R., Potter, C.S., Carragher, B. and Lyumkis, D., 2017.
Addressing preferred specimen orientation in single-particle cryo-EM through tilting. Nature
methods, 14(8), pp.793-796. \url{https://www.nature.com/articles/nmeth.4347}
	\end{enumerate}
	\item What is the 3DFSC? How does it relate to FSC and how does it diagnose preferred
orientation issues?
	\begin{itemize}
	\item {\bf COMPLETE BEFORE MEETING}. Let's analyze some read data with the 3DFSC, using a webserver \url{https://github.com/geoffwoollard/learn_cryoem_math/blob/master/cryoem-group-study-meeting/20200513_practical_3DFSC.pdf}
	\end{itemize}
\end{itemize}
\textendash{Solving the Problem}
\begin{itemize}
	\item What are the ways to mitigate preferred orientation?
	\item How does tilting mitigate preferred orientation? Pros and cons?
\end{itemize}

\textendash{Meeting Recording} \\
{\tiny \url{https://utoronto.zoom.us/rec/share/ly-8b5zAE_Ms0GDQq6nk0DMqpsVmF0rHH9hpc64oGlue9Cl070ck5VlzD2YIFaBF.6eDHwBlsaOavK_7c}}

Access Passcode: \texttt{1h5CGMTg\%6}

%	\item Tan, Y. Z., Baldwin, P. R., Davis, J. H., Williamson, J. R., Potter, C. S., Carragher, B., \& Lyumkis, D. (2017). Addressing preferred specimen orientation in single-particle cryo-EM through tilting. Nature Methods, 14(8), 793-796. http://doi.org/10.1038/nmeth.4347
	
%\textendash {Questions}
%\begin{enumerate}
%	\item Why might some samples have a preferred orientation? What can be done besides tilting to mitigate this?
%	\item How uniform are typical cryoEM datasets? How much preferred orientations can be tolerated before there are major problems? 
%	\item Diagnosing preferred orientations a 3D angular distribution plot assumes that the alignments are estimated well. However, preferred orientations may be so problematic that a 3D reconstruction cannot be determined. If this is the case, what could we do to check if preferred orientations were indeed a problem?
%	\item Tilted and non-tiled data can be combined, and the ratio optimized. Propose a concrete plan to optimize this ratio. Use specific numbers of particles, tilts - what exactly is being optimized?
%\end{enumerate}

\subsection{20 May 2021 - Sharpen - Master class: preferred orientation. Philip Baldwin (Salk Institute for Biological Studies \& Baylor College of Medicine).}
\textendash{Learning objectives}
\begin{enumerate}
	\item Understand how sampling quantitatively affects resolution
	\item Have a visual intuition for what parts of Fourier space are missing in different preferred orientation scenarios.
	\item Understand the assumptions underling the sampling compensation factor (SCF), when they hold and when they break down.
\end{enumerate}
\textendash{Pre-reading \& Questions}
\begin{itemize}
	\item Baldwin, P.R. and Lyumkis, D., 2020. Non-uniformity of projection distributions attenuates resolution
in Cryo-EM. Progress in biophysics and molecular biology, 150, pp.160-183. \url{https://www.biorxiv.org/content/10.1101/635938v2}. 
{\it Note that biorxiv preprint and Progress in biophysics and molecular biology article number the equations referred to below identically.}
	\begin{enumerate}
		\item What is the Spectral signal-to-noise ratio (SSNR) and how is it related to the FSC? For typical FSC curves (sigmoid like), what would the SSNR look like? 
		\item Why does the SSNR not increase with more data when there is preferred orientations, and what effect this has on the FSC?
		\item Explain the assumptions going from Eq. 2.10 to 2.11 in your own words.
		\item During iterative refinement, how would you compute the pieces that go into Eq 2.13: $E^2(k),  \langle|X|^2\rangle, \langle|N(\vec k)|^2\rangle, \langle \frac{1}{Sp(\vec k)}\rangle$
		\item What mathematical steps did you not understand, either mathematically, or the physical assumptions justifying certain mathematical leaps?
	\end{enumerate}
	\item 3DFSC webserver \url{https://github.com/geoffwoollard/learn_cryoem_math/blob/master/cryoem-group-study-meeting/20200513_practical_3DFSC.pdf}
	\begin{enumerate}
		\item Submit your own job to the 3DFSC webserver (halfmaps, full map, pixel size). Open the .cmd file in Chimera and have it ready for the meeting.
	\end{enumerate}
	\item Baldwin, P. R., \& Lyumkis, D. (2021). Tools for visualizing and analyzing Fourier space sampling in Cryo-EM. Progress in Biophysics and Molecular Biology, 160, 53-65. \url{http://doi.org/10.1016/j.pbiomolbio.2020.06.003}
\end{itemize}

\subsection{27 May 2021 - Blur - Fourier transforms and reciprocal space for the beginner (Grant Jensen's course)}
\textendash{Pre-reading}
\begin{enumerate}
%	\item \href{https://youtu.be/-EAQm8wgLbc}{Part 2: Fourier Transforms for Beginners - G. Jensen} (1 min)
%	\item \href{https://youtu.be/OESy_ltOCvI}{Part 2: 1-D Sine Waves and Their Sums - G. Jensen} (33 min)
%	\item \href{https://youtu.be/IUaqeoMK5y4}{Part 2: 1-D Reciprocal Space - G. Jensen} (20 min)
	\item \href{https://youtu.be/nyk75ufbP74}{Part 2: 2-D Waves and Images - G. Jensen} (19 min)
	\item \href{https://youtu.be/fEyLh9HqsWU}{Part 2: 2-D Transforms and Filters - G. Jensen} (33 min)
%	\item \href{https://youtu.be/INtehLmqfmA}{Part 2: 3-D Waves and Transforms - G. Jensen} (13 min)
%	\item \href{https://youtu.be/MQm6ZP1F6ms}{Part 2: Convolution and Cross-Correlation - G. Jensen} (15 min)
	\end{enumerate}
\textendash {Questions}
\begin{enumerate}
	\item In \href{https://youtu.be/nyk75ufbP74}{Part 2: 2-D Waves and Images}, Grant Jensen shows some visualizations of 2D waves. He enters values for h, k, amplitude, and phase, and a wave appears on the screen. Let's say we wanted to "spin a wave around" the origin. For instance make 360 waves that were each rotated by 1 degree. What values would we enter?
	\item If we took the Fourier transform of one of the 2D waves in \href{https://youtu.be/nyk75ufbP74}{Part 2: 2-D Waves and Images}, what would it look like? What happens as the wave is translated?
	\item Sometimes people see "water rings" in the power spectra during CTF estimation. How exactly does it arise, and why is it a ring? What makes it more or less bright? What controls the ring's width/spread?
	\item How would you make a checker-board-like pattern (e.g. 8x8) with Jensen's wave program?
	\item If we Fourier crop (hard low pass filter) a 3D map (256 pixels, $1 \AA$ / pixel) at $8 \AA$, how many "waves" go into the map?
	\item Around 15:20 in \href{https://youtu.be/fEyLh9HqsWU?t=920s}{Part 2: 2-D Transforms and Filters} Jensen remarks that the location of pixels in Fourier space is perpendicular to the location of the crest of the waves in real space. If we are on the scope and take a live FFT of a gridded pattern (pieces of a lattice, rotated in all sorts of directions), we see a ring. What happens to the ring in the FFT when we translate the stage in one direction.
	\item Play around a coding notebook on the discrete Fourier transform I made here: \url{https://github.com/geoffwoollard/learn_cryoem_math/blob/98cce6f1142af0aea8647f7a488d899f7336792d/nb/dft_sandbox.ipynb}
\end{enumerate}
\textendash {Discussion group leaders}
\begin{itemize}
	\item Arunabh Athreya 
\end{itemize}

\subsection{3 June May 2021 - Blur/Sharpen - Office Hours}
\textendash{Pre-reading}
\begin{enumerate}
	\item \href{https://youtu.be/ehXqLnV_Jr4}{S2C2 CryoEM Image Processing Workshop: Day 2 Part 1} (3h:20m)
%	\item \href{https://youtu.be/HXIipHORI2Y}{S2C2 CryoEM Image Processing Workshop: Day 2 Part 2} (3h:40m)
	\end{enumerate}
\textendash {Questions}
\begin{enumerate}
	\item Some data processing algorithms have many parameters, and it can be difficult to understand what exactly they are all doing, which ones go together, and when you would use them, how they would change things, and when you'd know that things are working. Find some parameters in a data processing algorithm that you don't understand, and bring it to the office hours to discuss. If you have time, run some jobs by changing a few parameters.
	\item Some people would like to know more about how an cryoEM algorithm works, but can't follow the original paper describing. Bring some results you've generated from some cryoEM data processing algorithm to discuss.
	\item Some data processing algorithms generate intermediate plots. If you know how to interpret these plots, they can help diagnose if the algorithm it working how you think it should be. Bring some plots that you have difficulty understanding to the office hours to discuss.  
	\item Some cryoEM discussion forums have "power users" that really understand things thoroughly. If you can follow how they are thinking, you can learn a lot, trouble shoot your data processing problems, and move your projects along. Find some posts and threads with these power users, and bring them to the office hours to discuss.
\end{enumerate}

\subsection{10 June 2021 - Blur - Fourier transforms and reciprocal space for the beginner (Grant Jensen's course)}
\textendash{Pre-reading}
\begin{enumerate}
%	\item \href{https://youtu.be/-EAQm8wgLbc}{Part 2: Fourier Transforms for Beginners - G. Jensen} (1 min)
%	\item \href{https://youtu.be/OESy_ltOCvI}{Part 2: 1-D Sine Waves and Their Sums - G. Jensen} (33 min)
%	\item \href{https://youtu.be/IUaqeoMK5y4}{Part 2: 1-D Reciprocal Space - G. Jensen} (20 min)
%	\item \href{https://youtu.be/nyk75ufbP74}{Part 2: 2-D Waves and Images - G. Jensen} (19 min)
%	\item \href{https://youtu.be/fEyLh9HqsWU}{Part 2: 2-D Transforms and Filters - G. Jensen} (33 min)
	\item \href{https://youtu.be/INtehLmqfmA}{Part 2: 3-D Waves and Transforms - G. Jensen} (13 min)
	\item \href{https://youtu.be/MQm6ZP1F6ms}{Part 2: Convolution and Cross-Correlation - G. Jensen} (15 min)
	\end{enumerate}
\textendash {Questions}
\begin{enumerate}
	\item The Guinier plot fits a slope and reports a B factor. What is on the y-axis, and what is on the x-axis. How do you compute this from a map? 
	\item Each data point from the Fourier shell correlation curve arises from a comparison of half maps at a given resolution shell. What does a Fourier shell look like in real space? How can making artificially inflate the FSC?
	\item Play with a coding notebook I made here: \url{https://github.com/geoffwoollard/learn_cryoem_math/blob/0d23f7b914ef2bfbcc616e05d2dbc0fa27eaf38d/nb/guinier_fsc_sharpen.ipynb}
\end{enumerate}

\subsection{17 June 2021 - blur - Glaeser, Nogales \& Chiu (GNC), 1 Introduction and Overview (1.1, 1.2)}
\textendash {Pre-reading}
\begin{itemize}
	\item Nogales, Eva. (2021). {\it "Section 1.1 Visualizing biological molecules to understand life's principles} in Glaeser, R. M., Nogales, E., \& Chiu, W. (Eds.). (2021). Single-particle Cryo-EM of Biological Macromolecules. IOP Publishing. http://doi.org/10.1088/978-0-7503-3039-8 (abbreviated as GNC)
	\item Glaeser, Robert. (2021). {\it "Section 1.2 Recovery of 3D structures from images of weak-phase objects"} in GNC
\end{itemize}
\textendash {Questions}
\begin{enumerate}
	\item Nogales mentions several unique capabilities of cryo-EM 
	\begin{enumerate}
		\item samples not amenable to other structural biology techniques (polymers and viruses, integral membrane proteins),
		\item samples where that have properties that makes them amenable to cryoEM (large assemblies)
		\item samples with a challenge that poses no problem to cryoEM: scarce samples
		\item areas where data processing can help: compositionally heterogeneous samples, conformationally complex samples
	\end{enumerate}
	 Discuss some interesting examples that touch on these areas (e.g. work out of your lab, interesting projects you've come across).
	\item Structural biology techniques (CryoEM, NMR, SAXS, X-ray crystallography, etc) can be used on the same system to answer different but intertwined questions. Mention some research projects that used multiple structural biology projects, and what biological insight each technique provided.
	\item Are there any themes/topics/breakthroughs that you thought should have been included in Nogales overview of cryoEM? 
	\item An electrons phase changes when it passes through the sample. What is the physical cause of these phase modulations? What would properties of the sample would make the phase modulation be higher or lower?
	\item Eq. 1.6 in Glaeser's chapter backs up an approximation of the image wave function that preserved linearity. What exactly is $\epsilon$ and why is it small? Look at Eqs. 1.2 and 1.3. What physical assumptions does this underly. 
	\item For the values in equation 1.3 and 1.4, what can we control experimentally and what is just a physical constant (intrinsic material property of atoms, etc).
%	\item Eq. 1.4 is circularly symmetric. The CTF is often plotted in 1D
	\item Look up some reference(s) from the chapter in an area that interests you. What did you learn. If you were writing the chapter, would you have used this reference differently?
	\item What sentences in Glaeser's chapter do you not understand? Take note of this and ask for clarification in the meeting or in the Slack group.
\end{enumerate}

%\textendash {Assignment/homework}
%\begin{itemize}
%	\item 
%\end{itemize}

\subsection{24 June 2021 - blur - Glaeser, Nogales \& Chiu (GNC), 2 Sample Preparation Introduction and Overview (2.2 - 2.5)}
\textendash {Pre-reading}
\begin{itemize}
	\item GNC Chapter 2 Sample Preparation Introduction and Overview (sections 2.2 - 2.5)
\end{itemize}
\textendash {Questions}
\begin{enumerate}
	\item Section 2.2: Do you think negative stain is overused or underused among the people you work with? How do you use it?
	\item Section 2.2: Write down the new things you learned about negative stain from this chapter in an email and send it to your cryoEM friends who would appreciate it.
	\item Section 2.3: Did you ever use lacey grids, or self-wicking grids? How did they compare with holey grids?
	\item Section 2.3: How did you go about optimizing glow discharge? What did you try and what did you look for? How reproducible were the results? Have you done something fancy with treatments of different gasses?
	\item Section 2.3: How reproducible is vitrifying for you? Do vitrifying devices behave differently (see Fig 2.6)? Have you noticed any patterns?
	\item Section 2.4: At what resolutions does correction for the curvature of the Ewald sphere help? What does this have to do with particle size?
	\item Section 2.4: What weird things have you seen in micrographs, and what do you think is causing them? How could you do some measurements on the microscope to track down what is physically happening?
	\item Section 2.5: What causes the different behaviour at the support water interface (SWI) versus the air water interface (AWI)? Why do proteins not unfold as much at the SWI versus the AWI?
	\item Section 2.5: In Fig 2.10 the edge of the hole is a few lengths the size of the specimen. Is this to scale? How thick is the holey pattern typically?
	\item Section 2.5: Since optimization of sample preparation and vitrification conditions is sample dependent, it's hard to develop heuristic "rules-of-thumb". However, perhaps overall strategies can be generalized? What did you try, and in what order? How long did you spend optimizing certain aspects, and what results helped you to move forward?
	\item Section 2.5: Do you have any experience with the devices mentioned in "Table 2.2: Cryo-EM sample preparation devices currently under development"?
\end{enumerate}

\subsection{3 June May 2021 - Blur/Sharpen - Office Hours}
\textendash{Pre-reading}
\begin{enumerate}
	\item \href{https://youtu.be/HXIipHORI2Y}{S2C2 CryoEM Image Processing Workshop: Day 2 Part 2} (3h:40m)
	\end{enumerate}
\textendash {Questions}
\begin{enumerate}
	\item Some people would like to know more about how an cryoEM algorithm works, but can't follow the original paper describing. Bring some results you've generated from some cryoEM data processing algorithm to discuss.
	\item Some cryoEM discussion forums have "power users" that really understand things thoroughly. If you can follow how they are thinking, you can learn a lot, trouble shoot your data processing problems, and move your projects along. Find some posts and threads with these power users, and bring them to the office hours to discuss.
\end{enumerate}

\subsection{8 July 2021 - sharpen - Journal club: multisclice. Ben Himes (HHMI Janelia Research Campus)}
\textendash {Pre-reading}
\begin{itemize}
	\item Himes, B. A., \& Grigorieff, N. (2021). Cryo-TEM simulations of amorphous radiation-sensitive samples using multislice wave propagation. BioRxiv. 
	\\ \url{http://doi.org/10.1101/2021.02.19.431636}
	\\ \url{https://www.biorxiv.org/content/10.1101/2021.02.19.431636v2}
\end{itemize}
\textendash {Questions}
\begin{enumerate}
	\item What are the extensions of this paper compared to past treatments?
	\item What is not modelled?
	\item What are the computational bottlenecks in simulating images using multislice?
	\item Out of all the sources of noise, which ones dominate? How can this be tested/quantified?
	\item Why is exact quantitative agreement important (proportionality constant equal to unity)?
	\item Who is interested in using multislice simulators and why?
	\item Was there any terms you did not understand? (bremsstrahlung, plasmon, etc)
	\item Draw a picture illustrating the equations (especially Eqs. 2, 3 (the first one on line 118), 8, 10)
\end{enumerate}

\textendash{Meeting Recording} \\
{\tiny \url{https://utoronto.zoom.us/rec/share/_yXLHqvspeJ5iUGW5FHbTELQs6P9reUzpJiwM-tZK7PB6OAdslPXGac-y4eJhzJh.LvVIo5bYXXGzL40C}}
Access Passcode: \texttt{x5wP8\$3\$?j}

\subsection{15 July 2021 - Sharpen - 3DVA Part I}
\textendash{Pre-reading}
\begin{enumerate}
	\item Tutorial: 3D Variability Analysis (Part One) \url{https://guide.cryosparc.com/processing-data/tutorials-and-case-studies/tutorial-3d-variability-analysis-part-one}
	\item Sorzano, C. O. S., \& Carazo, J. M. (2021). Principal component analysis is limited to low-resolution analysis in cryoEM. Acta Crystallographica Section D Structural Biology, 77(6), 835?839. http://doi.org/10.1107/s2059798321002291
	\item \href{https://youtu.be/0O781Od1z_E}{Webinar: Resolving flexibility and heterogeneity with 3D Variability Analysis} (90 min video).
\end{enumerate}
\textendash {Questions}
\begin{enumerate}
	\item What is an eigenvolume?
	\item Look at "Eigenimage1" in Fig 1a of Sorzano \& Carazo 2021. Is it representative of any "particle" in the dataset?
	\item Prior to 3DVA other groups have published similar PCA approaches. Have you come across any? Are they widely known in the cryoEM community? 
	\item 3DVA webinar (8 min mark): what is a probabilistic graphical model?
	\item What are the inputs to 3DVA? 
	\item 3DVA webinar (20 min mark): Go through every notational aspect of the mathematical equation of the linear subspace model? What does everything refer to (all variables, subscripts, etc). In what sense is the linear subspace model linear?
	\item 3DVA webinar (27 min mark): Is there any way to distinguish among possible causes of the heterogeneity with 3DVA: motion, dissociation, ligand binding, floating junk?
	\item How could one recognize junk in latent variable space?
	\item What can and can't be concluded from the three modes shown for the GPCR in the 3DVA webinar (45 min mark)
\end{enumerate}

\textendash{Meeting Recording} \\
{\tiny \url{https://utoronto.zoom.us/rec/share/jGlUF-wjDSeI0HLMhJRk6Vo5eyd5MBZmwK2l_s0czPKwgEqX54LZNZCxoNM6yXk.3anCJWsjnaXUb9ld}}
Access Passcode: \texttt{95Qj5jC\$pW}

\subsection{22 July 2021 - Sharpen - 3DVA Part II}
\textendash{Pre-reading}
\begin{enumerate}
	\item Tutorial: 3D Variability Analysis (Part Two) \url{https://guide.cryosparc.com/processing-data/tutorials-and-case-studies/tutorial-3d-variability-analysis-part-two}
\end{enumerate}
%\textendash {Questions}
%\begin{enumerate}
%	\item Install 
%\end{enumerate}
\textendash {Discussion group leaders}
\begin{itemize}
	\item Yuichiro Takagi
\end{itemize}


\subsection{29 July 2021 - blur - Glaeser, Nogales \& Chiu (GNC), 3 Data Collection (3.2,3.3)}
\textendash {Pre-reading}
\begin{itemize}
	\item Rubinstein, John. (2021). 3.2 Radiation damage in cryo-EM in GNC Chapter 3 Data Collection.
	\item Cheng, Anchi. (2021). 3.3 Low-dose protocols for recording images in GNC Chapter 3 Data Collection. 
\end{itemize}
\textendash {Questions}
\begin{enumerate}
	\item Section 3.2.1: How much energy in eV is needed to break a covalent bond?
	\item Section 3.2.2: How much energy on average does each each inelastic interaction deposit into the specimen?
	\item Section 3.2.2: "Each informative elastic event is accompanied by the deposition of" how much "energy into the specimen"? Where exactly does this number come from?
	\item Section 3.2.2: This chapter cites (Henderson, 1995) for the energy deposited per inelastic event, and the number can be found in Table 2 of that paper. What sorts of experiment does this number come from. Under what conditions is it fixed? When would you expect this value to change?
	\begin{itemize}
		\item Henderson, R. (1995). The Potential and Limitations of Neutrons, Electrons and X-Rays for Atomic Resolution Microscopy of Unstained Biological Molecules. Quarterly Reviews of Biophysics, 28(2), 171-193. \url{http://doi.org/10.1017/S003358350000305X}
	\end{itemize}
	\item Section 3.2.2: What is an intuitive mental picture for the following fact: "Both inelastic and elastic interactions have higher cross sections at lower energies, so that more elastic scattering, more inelastic scattering, and more radiation damage occur for the same electron exposure with lower-energy electron microscopes"?
	\item Section 3.2.3: Distinguish primary, secondary, and tertiary radiation damage.
	\item Section 3.2.3: How could the local environment of the specimen influence radiation damage?
	\item Section 3.2.3: Why is "single-particle cryo-EM at liquid-nitrogen temperature is now generally accepted to be superior to cryo-EM at liquid-helium temperature"?
	\item Section 3.2.4: What is the advantage of working with a 100 kV scope, versus at 200 or 300 kV? 
	\item Section 3.2.4: What sample and data processing factors are called for at lower energies?
	\item Section 3.2.4: Why aren't people getting good structures at 100 keV now? What would enable 100 keV energies to yield high resolution structures?
	\item Section 3.2.4: To have roughly the same "shape" of CTF at low resolution (which affects the visibility of particles), how should the defocus be adjusted with the beam energy?
\end{enumerate}

\subsection{12 Aug 2021 - blur - Glaeser, Nogales \& Chiu (GNC), 3 Data Collection (3.4)}
\textendash {Pre-reading}
\begin{itemize}
	\item Danev, Radostin. (2021). 3.4 Practical considerations: defocus, stigmation, coma-free illumination, and phase plates.
\end{itemize}
\textendash{Meeting Recording} \\
{\tiny \url{https://utoronto.zoom.us/rec/share/8Tqw3X99DJKqu5B4QcbfKkfUFW0zUlQJ6j9\_x9vVuidyPhcutzUIxBlaLz2ZQLto\.s2Y95VFXqOsiPnQh}} \\
Access Passcode: \texttt{\&vX6zAi1@f}

\subsection{19 Aug 2021 - blur - Glaeser, Nogales \& Chiu (GNC), 3 Data Processing (4.3)}
\textendash {Pre-reading}
\begin{itemize}
	\item Himes, Benjamin A, and Nikolaus Grigorieff. (2021). 4.3 CTF estimation and image correction (restoration).
\end{itemize}
\textendash {Questions}
\begin{enumerate}
	\item In what sense was the CTF applied to the image of Einstein in Fig 4.6B?
	\item How does w in Eq 4.2 affect the 1D plot of the CTF?
	\item Plug the Scherzer defocus $(\Delta Z = - \sqrt{C_s\lambda})$ into the wave aberration of the CTF in Eq 4.3 
	$(\gamma = 2\pi [ \frac{C_s}{4}\lambda^3 s^4 - \frac{\Delta Z}{2}\lambda s^2 ])$. Can you rework things to make $\gamma = 2 \pi [x^2 + x]$? If so, what is $x$? Is the Scherzer defocus under or over focused?
	\item 4.3.1: The authors mention {\it "The sinusoidal modulations, sometimes referred to as Thon rings, form a characteristic pattern of rings or ellipses observed in computer-generated power spectra. They can be used to determine the defocus and astigmatism to within about 100 $\AA$ [...]. Once a reconstruction has been determined, the defocus parameters can be further refined and other, less significant errors can be measured and corrected (section 4.8). A full correction is therefore an iterative process that starts with an initial estimation of defocus and astigmatism from the electron micrographs themselves, without reference to a 3D reconstruction.}

In your experience with data processing, how much are par particle defoci adjusted vs whole micrograph defocus? How much variability in defocus is there in a micrograph? What else could cause an estimate of defocus to be different vs the physical height of the specimen? How precise and reproducible are CTF estimation algorithms?

	\item 4.3.1: Have you noticed CTF estimation failing? In what cases? Have you been able to fix things by changing default parameters?
	\item 4.3.1: McMullan et al (2015) estimated that "water molecules, each with a molecular mass of 18 Da, move an RMS distance of $ \sim 1 \AA$ for each $e^-/\AA^2$" What would happen if water molecules moved more or less?
		\begin{itemize}
			\item McMullan, G., Vinothkumar, K. R., \& Henderson, R. (2015). Thon rings from amorphous ice and implications of beam-induced Brownian motion in single particle electron cryo-microscopy. Ultramicroscopy, 158, 26--32. \url{http://doi.org/10.1016/j.ultramic.2015.05.017}
		\end{itemize}
	\item 4.3.2: Consider Eqs. 4.7-4.9. What happens at the spatial frequencies where the CTF is zero? How does the image behave at spatial frequencies very close to $(s_x,s_y)$, where the CTF is non zero? How does this change when the SNR is more or less? Useful link: \url{http://www.bio.brandeis.edu/~shaikh/lab/ctf.htm}.
		\begin{enumerate}
			\item Consider the behaviour at one point  $(s_x^\prime,s_y^\prime)$
			\item Assume there are four images. Thus $M=4$, and our measurements at the point we are interested in are: $X_1(s_x^\prime,s_y^\prime), X_2(s_x^\prime,s_y^\prime), X_3(s_x^\prime,s_y^\prime), X_4(s_x^\prime,s_y^\prime)$, where each $X_i = CTF_i \cdot F_i + N_i$
			\item Assume that the CTFs are: $ CTF_1(s_x^\prime,s_y^\prime)=0.1, CTF_2(s_x^\prime,s_y^\prime)=0.2, CTF_3(s_x^\prime,s_y^\prime)=-0.1, CTF_4(s_x^\prime,s_y^\prime)=-0.2 $
			\item Assume that the (random) noise at each image at the $(s_x^\prime,s_y^\prime)$ is: $N_1(s_x^\prime,s_y^\prime)=1, N_2(s_x^\prime,s_y^\prime)=1, N_3(s_x^\prime,s_y^\prime)=-1, N_4(s_x^\prime,s_y^\prime)=1 $ 
			\item Assume the signal is the same (because the images are aligned): $F_1(s_x^\prime,s_y^\prime)= F_2(s_x^\prime,s_y^\prime)=F_3(s_x^\prime,s_y^\prime)=F_4(s_x^\prime,s_y^\prime)=1 $ 
			\item Work out the numerator in Eq. 4.9 by multiplying the CTF by the measurement
			\item Work out the denomenator in Eq. 4.9 by multiplying the CTF by the measurement
			\item Using Eq. 4.7 work out $|F(s_x^\prime,s_y^\prime)|$, assuming $SNR(s_x^\prime,s_y^\prime)=1$.
			\item Repeat this calculation, but doubling the values of the CTFs: $0.2, 0.4, -0.2, -0.4$. What happens to our estimate of F as the magnitude of the CTFs are larger?
			\item Repeat this calculation, with CTFs all equal to $1$. As M gets larger, what will happen to $\sum_i CTF_i^2$ vs $1/SNR$?
		\end{enumerate}
	\item 4.3.2 Consider Eq. 4.10 $D_W =D_P +2\frac{\lambda}{d}\Delta Z.$ The text works out a calculation: $D_W = 200 \AA + 2 \frac{0.0224 \AA}{2 \AA}10000 \AA = 200 \AA + 224 \AA \sim 400 \AA$. Calculate the limiting resolution d for box sizes and specimen diameters you are working with. Do you think box size may be a limiting factor among practitioners? (Note: reference wavelengths of fast electrons for convenience: 0.0388 $\AA$ 100 keV, 0.0274 $\AA$ at 200 keV, and 0.0224 $\AA$ at 300 keV).
\end{enumerate}

\subsection{26 Aug 2021 - blur/sharpen - Guest: Basil Greber (chapter author) Glaeser, Nogales \& Chiu (GNC), 4 Data Processing (4.4)}
\textendash {Pre-reading}
\begin{itemize}
	\item Basil J Greber. (2021). 4.4 Merging data from structurally homogeneous subsets.
\end{itemize}
\textendash {Questions}
\begin{enumerate}
	\item 4.4.1.2: why does "retrieving all the information needed to fill the 3D Fourier transform requires more views for a larger particle"?
	\item 4.4.1.2: How exactly do the effects of better alignment estimation and more views "cancel each other out"? See Eq. A 11 in Hederson, 1995. %TODO work through appendix
	\item 4.4.1.3: How many particles of apoferritin are needed to reach $3 \ \AA$? What's the record for the least amount of particles? How does this compare to theoretical estimates?
	\item 4.4.1.3: Why are low molecular weight specimens difficult? What parameters should be tweaked inside of algorithms to make things work for them, and why?
	\item 4.4.1.4: Henderson-Rosenthal plots are $\log$[particle number] vs $1/d^2$, while ResLog plots are vs $1/d$. Do they address the same question? Why the $1/d^2$ vs $1/d$?
	\item 4.4.2.1: What information about the particle alignments (e.g. from 2D classification) is used in 3D reconstruction (in various software packages)? Note that this may not be widely known, require looking at the original publications, and reading the current source code, if available.
	\item 4.4.2.2: Do you have any experience using packages that implement algebraic methods (ART, SIRT), such as IMAGIC? What was your experience?
	\item Have you tried out the same dataset in different data processing software? What was your experience?
\end{enumerate}
\textendash{Meeting Recording} \\
{\tiny \url{https://utoronto.zoom.us/rec/share/hvkUzMRMQNC7MR9QKulXTeDrvbf0loy6R492MEAm5MA2XkxyoGErGDnn2Pz5p27j.bzGSELDYYPDehKwz}} \\
Access Passcode: \texttt{Dm14.p3O8V}

\end{document}  